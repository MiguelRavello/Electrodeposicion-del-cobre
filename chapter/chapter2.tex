\chapter{FUNDAMENTOS TE\'ORICOS}
\section{GENERALIDADES DEL COBRE}

Fue uno de los primeros metales empleados por el hombre por encontrarse en su estado nativo. Actualmente la producci\'on de cobre se obtiene de minerales en forma de sulfuros: Calcopirita $(CuFeS_2)$, Calcosina $(Cu_2S)$, Covelita $(CuS)$, y minerales oxidados, como la Cuprita $(Cu_2O)$, la Tenorita $(CuO)$, etc.

\begin{table}[H]
\label{tabla1}
\begin{center}
\begin{tabular}{|c|c|}
\hline
S\'imbolo&Cu\\
\hline
Numero at\'omico&29\\
Peso at\'omico&63.5\\
Valencias&1.2\\
Densidad	&8.9\\
Punto de fusi\'on&$1083^oC$\\
Punto de ebullici\'on&$2595^oC$\\
\hline
\end{tabular}
\end{center}
\caption{Propiedades f\'isicas del cobre}
\end{table}


\subsection{USOS}

Los principales usos del cobre como producto terminado son:\\

Tuber\'ia de cobre: En la industria automotriz, refrigeraci\'on, agroindustria, industria de la construcci\'on, etc. \\
Las sales de cobre como el sulfato y oxicloruro de cobre se emplean como fungicida en agricultura y el oxido cuproso como base de ciertas pinturas.
Las aleaciones de cobre: Los latones son aleaciones de cobre con zinc, se utilizan para cartuchos de municiones, en la industria automotriz (en los radiadores), ferreter\'ia, accesorios para plomer\'ia, joyer\'ia de fantas\'ia, intercambiadores de calor, estuches para l\'apiz labial, polveras, etc. \\
En la industria de las comunicaciones y manufacturera: Por su elevada conductividad el\'ectrica se utiliza mayormente en la fabricaci\'on de conductores el\'ectricos (cables el\'ectricos).\\
Los bronces son aleaciones de cobre con esta\~no fundamentalmente, con adiciones de otros elementos en menor proporci\'on tales como plomo, n\'iquel, etc., as\'i tenemos:\\
\begin{itemize}
 \item Los bronces al esta\~no, se utilizan en la fabricaci\'on de bocinas, engranajes, partes y mecanismos de bombas, etc.
 \item Bronces al silicio, se utilizan en recipientes a presi\'on, construcci\'on marina de conductos hidr\'aulicos a presi\'on, etc.
 \item Los bronces al aluminio, se utilizan en engranajes, ejes motrices, piezas de bombas, etc.
 \item La Plata alemana (5 a 30\% n\'iquel, y 5 a 40\% zinc), se utiliza en resortes y contactos en equipos para tel\'efonos, equipo quir\'urgico y dental.
 \item Maillechort \'o metal para resistencias el\'ectricas con aleaci\'on cobre-n\'iquel.
\end{itemize}

\subsection{MINERALOG\'IA}
Las principales especies minerales del cobre se muestran a continuaci\'on:

\begin{table}[H]
\label{tabla2}
\begin{center}
\begin{tabular}{|c|c|c|c|}
\hline
\multirow{2}{2.5cm}{Zona mineralizada} & \multirow{2}{2.5cm}{Especie Mineral\'ogica}
 &\multirow{2}{5cm}{Composici\'on M\'as Frecuente Para Esta Especie} & \multirow{2}{2cm}{Cobre (\%)} \\
 & & & \\
\hline 
\multirow{11}{2.5cm}{Zona oxidada secundaria} & \multirow{2}{2.5cm}{Malaquita} & \multirow{2}{5cm}{$CuCO_3.Cu(OH)_2$  \'o  $Cu_2CO_3(OH)_2$}&\multirow{2}{2cm}{\: \: \: 57.5}\\
 & & & \\
\cline{2-4}
 & \multirow{2}{2.5cm}{Azurita} & \multirow{2}{5cm}{$2CuCO_3.Cu(OH)_2$  \'o  $Cu_3(CO_3)_2(OH)_2$}&\multirow{2}{2cm}{\: \: \: 55.3}\\
 & & & \\
\cline{2-4}
 & \multirow{2}{2.5cm}{Crisocola} & \multirow{2}{5cm}{$CuO.SiO_2.H_2O$  \'o   $CuSiO_3. H_2O$}&\multirow{2}{2cm}{\: \: \: 36.2}\\
 & & & \\
 \cline{2-4}
 & \multirow{2}{2.5cm}{Branchita} & \multirow{2}{5cm}{$CuSO_4.3Cu(OH)_2$   \'o   $Cu_4SO_4(OH)_4$}&\multirow{2}{2cm}{\: \: \: 56.2}\\
 & & & \\
 \cline{2-4}
 &Cuprita&	$Cu_2O$&	88.8\\
 \cline{2-4} 
 &Tenorita&	$CuO$&	79.9\\
 \cline{2-4} 
 &Cooper Wad&	$CuMnO_2Fe$&	Variable\\
 \cline{2-4} 
 &Cooper Pitch&	$CuMn_8FeSiO_2$& 	Variable\\
 \cline{1-4}
\multirow{3}{2.5cm}{Zona de Enriquecimiento Secundario}& Calcosina&$Cu_2S$&79.9\\
 \cline{2-4}
 &Digenita&$Cu_9S_5$&78.1\\
 \cline{2-4}
 &Covelina&$CuS$&66.5\\
 \hline
\multirow{3}{2.5cm}{Zona Primaria}&Calcopirita&$CuFeS_2$&	34.6\\
 \cline{2-4}
 &Bornita&$Cu_5FeS_4$&63.3\\
 \cline{2-4}
 &Enargita&$Cu_3AsS_4$&48.4\\
 \hline
\end{tabular}
\end{center}
\caption{Principales especies minerales}
\end{table}

\subsection{PRODUCCI\'ON NACIONAL}
El Per\'u se encuentra como segundo productor a nivel mundial y latinoamericano en lo que a cobre se refiere, seg\'un se muestra en la tabla siguiente:
\begin{figure}[H]
 \centering
 \includegraphics[scale=0.5]{Imagenes/Vectorial/a.png}
 % a.png: 652x453 pixel, 72dpi, 23.00x15.98 cm, bb=0 0 652 453
 \caption{Per\'u-posici\'on de producci\'on minera 2011}
 \label{fig:1}
\end{figure}

\begin{figure}[H]
 \centering
 \includegraphics[scale=0.4]{Imagenes/Vectorial/b.png}
 % b.png: 998x436 pixel, 72dpi, 35.21x15.38 cm, bb=0 0 998 436
 \caption{Producci\'on de cobre en toneladas por pa\'is}
 \label{fig:2}
\end{figure}

Tambi\'en mostramos la producci\'on y reservas de cobre del Per\'u:
\begin{figure}[H]
 \centering
 \includegraphics[scale=0.5]{Imagenes/Vectorial/c.png}
 % c.png: 750x540 pixel, 72dpi, 26.46x19.05 cm, bb=0 0 750 540
 \caption{Producci\'on y reservas de cobre en el Per\'u al 2011}
 \label{fig:3}
\end{figure}

En el siguiente gr\'afico mostramos la evoluci\'on en producci\'on de cobre desde el a\~no 2000 hasta el a\~no 2011:
\begin{figure}[H]
 \centering
 \includegraphics[scale=0.4]{Imagenes/Vectorial/d.png}
 % d.png: 834x532 pixel, 72dpi, 29.42x18.77 cm, bb=0 0 834 532
 \caption{Producci\'on de cobre Y cotizaci\'on 2000-2011}
 \label{fig:4}
\end{figure}

\section{EL PROCESAMIENTO DE MINERALES DE COBRE} 
\paragraph{OBTENCI\'ON DEL COBRE A PARTIR DE SUS \'OXIDOS:}

Si se trata de una mena oxidada, la cuprita $(Cu_2O)$ o la Tenorita $(CuO)$ se le acondiciona en una pila (mont\'iculo de mineral) donde se le roc\'ia con una soluci\'on de \'acido sulf\'urico, en el proceso denominado de lixiviaci\'on, obteni\'endose en esta etapa la llamada soluci\'on rica o pre\~nada que consiste en agua, sulfato de cobre e impurezas. En la siguiente etapa, denominada de purificaci\'on, se somete, la soluci\'on rica, a la acci\'on de solventes org\'anicos, dejando en la soluci\'on \'acido sulf\'urico regenerado e impurezas, los que son retornados a las pilas. El electrolito cargado pasa a las celdas de electrodeposici\'on, en donde el electrolito se descompone por acci\'on de la corriente el\'ectrica, deposit\'andose el cobre en el c\'atodo de la celda, y obteni\'endose un cobre de alta pureza (99.999\%). El electrolito est\'eril vuelve a unirse al solvente cargado para tomar el cobre y renovar el ciclo.\\
En la naturaleza tambi\'en existen recursos minerales de cobre que se encuentran en la forma de \'oxidos, entre los m\'as abundantes se pueden mencionar:\\

\begin{tabular}{cc}
Malaquita&$CuCO_3.Cu(OH)_2$\\
Atacamita&$CuCl(OH)Cu(OH)_2$\\
Crisocola&$CuSiO_3.nH_2O$\\
Azurita&	$2CuCO_3.Cu(OH)_2$\\
Cuprita&	$Cu_2O$\\
Copper wad& \\
Copper Pitch& \\
\end{tabular}

\paragraph{OBTENCI\'ON DE COBRE A PARTIR DE LOS SULFUROS:}

Se realizan m\'ultiples etapas con los minerales como chancado, molienda y flotaci\'on obteniendo concentrados de cobre que luego ingresan a procesos de tostaci\'on, fundici\'on y refinaci\'on hasta obtener un cobre de pureza 99.99\%.\\
Hace 30 a\~nos atr\'as la producci\'on de cobre se deb\'ia casi en un 95\% a los minerales sulfurados de cobre, tales como:\\

\begin{tabular}{cc}
 Calcopirita&$CuFeS_2$\\
Bornita&$Cu_5FeS_4$\\
Calcocita&$Cu_2S$\\
Covelina&$CuS$\\
\end{tabular}
 \\
 
Que se explotan a trav\'es de procesos de flotaci\'on, seguido de procesos pirometal\'urgicos y electro refinaci\'on.\\ 
Los procesos aplicados a los s\'ulfuros de cobre no se aplican a los \'oxidos de cobre, porque se requieren altos requerimientos energ\'eticos, que implican altos costos operacionales que imposibilitan su aplicabilidad. Por otro lado, no exist\'ian procesos hidrometal\'urgicos eficientes que permitieran la explotaci\'on de minerales de cobre oxidados de leyes relativamente bajas, sin embargo, existían algunas faenas que operaban con minerales oxidados de cobre con leyes sobre el 2,5\% y aplicaban un proceso de lixiviaci\'on, seguido por un proceso de cementaci\'on. Para la peque\~na miner\'ia, la explotaci\'on de minerales oxidados de cobre estaba limitada a leyes sobre el 4 \% de cobre.\\

\section{LIXIVIACI\'ON}
Es una etapa fundamental en el proceso, que involucra la disoluci\'on del metal a recuperar desde una materia prima s\'olida, en una soluci\'on acuosa mediante la acci\'on de agentes qu\'imicos.\\
Esta transferencia del metal hacia la fase acuosa, permite la separaci\'on del metal contenida en la fase s\'olida de sus acompa\~nantes no solubles. En la hidrometalurgia del cobre dada su variedad de sustancias s\'olidas que contienen cobre factible de beneficiar por lixiviaci\'on, complican la extensi\'on de los fundamentos del sistema lixiviante (s\'olido-agente-extractante-m\'etodos).

\begin{figure}[H]
 \centering
 \includegraphics[scale=0.5]{Imagenes/Vectorial/e.png}
 % e.png: 1095x1394 pixel, 200dpi, 13.91x17.70 cm, bb=0 0 394 502
 \caption{Proceso t\'ipico de un sistema de lixiviaci\'on}
 \label{fig:e}
\end{figure}

Factores T\'ecnicos y Econ\'omicos involucrados en un an\'alisis de un proyecto de lixiviaci\'on.
\begin{itemize}
 \item Ley de la especie de inter\'es a recuperar
 \item Reservas de mineral
 \item Caracterizaci\'on mineral\'ogica y geol\'ogica
 \item Comportamiento metal\'urgico
 \item Capacidad de procesamiento
 \item Costos de operaci\'on y de capital
 \item Rentabilidad econ\'omica
\end{itemize}

A continuaci\'on mostramos c\'omo reaccionan algunos minerales de cobre a la lixiviaci\'on \'acida:

\begin{enumerate}
 \item Atacamita  $CuCl_23Cu(OH)_2$   (50.5\% Cu) (se lixivia f\'acilmente)
  \begin{equation}
   CuCl_23Cu(OH)_2 + 3H_2SO_4 \longrightarrow  3CuSO_4 + CuCl_2 + 6H_2O 
  \end{equation}
 \item Crisocola $CuSiO32H2O$      (36.1\% Cu) (silicato de estructura cristalina compleja, mientras m\'as compleja mas dif\'icil lixiviar)
 \begin{equation}
  CuSiO_32H_2O + H_2SO_4 \longrightarrow        CuSO_4 + SiO_2 + 3H_2O
 \end{equation}
 \item Calcosina: $Cu_2S$          (79.8\% Cu) (el 50\% se lixivia r\'apidamente)
 \begin{equation}
  \begin{align}
   Cu_2S + Fe_2(SO_4)_3 & \longrightarrow                     CuS + CuSO_4 + 2FeSO_4 \\ 
CuS + Fe_2(SO_4)_3  & \longrightarrow                     CuSO_4 + 2FeSO_4 +S \\
Cu_2S + 2Fe_2(SO_4)_3   &  \longrightarrow                 2CuSO_4 + 4FeSO_4 + S
  \end{align}
 \end{equation}
 \item Covelina: $CuS$            (66.4\% Cu) (de lixiviaci\'on lenta)
 \begin{equation}
  CuS + Fe_2(SO_4)_3  \longrightarrow        CuSO_4 + 2FeSO_4 + S 
 \end{equation}
 \item Bornita: Cu5FeS4      (63.3\% Cu)
 \begin{equation}
  Cu_5FeS_4 + 6Fe_2(SO_4)_3   \longrightarrow             5CuSO_4 + 13FeSO_4 + 4S 
 \end{equation}

\end{enumerate}
  
\subsection{TERMODIN\'AMICA}
La forma m\'as conveniente de representar la termodin\'amica de sistemas acuosos es en forma gr\'afica en los diagramas de Pourbaix o diagramas potencial-pH. Estos diagramas son ampliamente utilizados por los profesionales, por cuanto permiten visualizar posibilidades de reacciones sin tener que recurrir al c\'alculo termodin\'amico para los fen\'omenos que ocurren en medio acuoso.
Una importante restricci\'on en la aplicaci\'on pr\'actica de los diagramas termodin\'amicos, es que predicen tendencias a que ocurran fen\'omenos, pero no la velocidad con que \'estos puedan ocurrir. En la pr\'actica las velocidades de reacci\'on pueden variar desde valores tan altos que son controlados por limitaciones en la transferencia de masa, a valores tan bajos que se requieren per\'iodos geol\'ogicos para observar en forma directa el fen\'omeno. La cin\'etica extremadamente lenta en algunas reacciones conduce a que algunas fases s\'olidas existan en condiciones fuera de su rango de estabilidad termodin\'amica o que fases s\'olidas no se formen en condiciones termodin\'amicas favorables y lo hagan otras en su lugar (fases metaestables) (ejemplo: precipitaci\'on de hidr\'oxido de hierro). En este caso, es a veces \'util utilizar diagramas Eh-pH modificados que consideren las fases metaestables.

\begin{figure}[H]
 \centering
 \includegraphics[scale=0.5]{Imagenes/Vectorial/diag.png}
 % diag.png: 689x672 pixel, 96dpi, 18.23x17.78 cm, bb=0 0 517 504
 \caption{Diagrama Eh-pH del sistema cobre-agua(T=$25^oC$, actividades = 1 ( [iones]=1 ) , P = 1 atm )}
 \label{fig:f}
\end{figure}

\paragraph{INTERPRETACI\'ON}
\begin{itemize}
 \item La disoluci\'on de los \'oxidos simples de cobre es termodin\'amicamente posible en el dominio \'acido y en presencia de oxidantes. La tenorita $(CuO)$ s\'olo necesita condiciones de pH, mientras que en esas condiciones, la cuprita $(Cu_2O)$ necesita adem\'as la presencia de un agente oxidante (iones $Fe^{3+}$, $O_2$, u otros ).\\
 Las reacciones son:
 \begin{equation}
  \begin{align}
   CuO + 2 H^+	& \Longleftrightarrow	Cu_2+ + H_2O\\
   Cu_2O + 2H^+& \Longleftrightarrow 2Cu^{2+} + H_2O + 2e^-\\
   Ox. + 2 e^- & \Longleftrightarrow Red.\\
   Cu_2O + 2 H^+ + Ox. & \Longleftrightarrow	2 Cu_2+ + Red. + H_2O
  \end{align}
 \end{equation}
 en que Ox. representa un agente oxidante cualquiera.
 \item En forma inversa, al estar el $Cu^{2+}$ en soluci\'on, y para poder permanecer en ella, necesita de una cierta acidez libre, evit\'andose de esta manera su posterior precipitaci\'on a pH >4.
 \item A trav\'es de todo el rango de pH, el cobre met\'alico es termodin\'amicamente estable estando en contacto con agua, pero en ausencia de oxigeno (u otro oxidante).
 \item La precipitaci\'on electrol\'itica se puede realizar aplicando al c\'atodo un potencial inferior a 0.34 V. De esta forma el cobre $Cu^{2+}$ se reduce en el c\'atodo de acuerdo a:
 \begin{equation}
  Cu^{2+} + 2e^- \Longleftrightarrow Cu^o  \text{(c\'atodo)}
 \end{equation}
\end{itemize}

\subsection{CIN\'ETICA}
El hecho de que una reacci\'on sea termodin\'amicamente posible $(\Delta G<0)$, no es suficiente para predecir si la reacci\'on va a pasar en una escala de tiempo razonable.  Eso  depende  de  la  cin\'etica  de  la  reacci\'on.  Este  factor  es  muy importante para la concepci\'on y la evaluaci\'on de la rentabilidad econ\'omica de todos los procesos hidrometalurgicos. Tambi\'en en las plantas en operaci\'on, optimizar la cin\'etica resulta generalmente en un mejoramiento del proceso.
Por ejemplo, la lixiviaci\'on de la calcopirita y otros sulfuros con sulfato f\'errico es termodin\'amicamente factible (Existen discrepancias respecto a la reacci\'on con sulfato f\'errico, tambi\'en ocurre la reacci\'on $CuFeS_2 + 4 Fe^{3+} \rightarrow Cu^{2+} + 5 Fe^{3+} +2S^o):$
\begin{equation}
 \begin{align}
CuFeS_2 + 8 H_2O& \rightarrow Cu^{2+} + Fe^{3+} + 2 SO_4^{2-} + 17 e^- + 16 H^+\\
17 Fe^{+3} +17 e^-& \rightarrow 17 Fe^{2+}\\
               &\Delta G<0\\
CuFeS_2 + 16 Fe^{+3} + 8H_2O & \rightarrow Cu^{2+} + 17 Fe^{2+} + 2 SO_4^{2-} + 16 H^+
 \end{align}
\end{equation}

Pero en la pr\'actica, despu\'es de 100 días de lixiviaci\'on, s\'olo se alcanza a poner en soluci\'on un 5\% de la calcopirita, 50\% de calcosina y 80\% de covelina.
\begin{itemize}
 \item Reacci\'on Homog\'enea:\\
 Reacci\'on qu\'imica u electroqu\'imica en la cual todos los productos y reactantes pertenecen a una sola y misma fase.\\
 Ejemplo:
 \begin{equation}
   MnO_4^- + 5 Fe^{2+} + 8 H^+ \Longleftrightarrow Mn^{2+} + 5 Fe^{3+} + 4 H_2O
 \end{equation}
 \item Reacci\'on Heterog\'enea:\\
 Una reacci\'on es heterog\'enea si tiene lugar en dos o m\'as fases.\\
 Ejemplo:
 \begin{equation}
  NaCl \text{(s\'olido)} \Longleftrightarrow Na^+ + Cl^- \text{(soluci\'on)}
 \end{equation}
\end{itemize}
Las etapas principales de una reacci\'on son:
\begin{enumerate}
 \item Transporte de masa de los reactantes gaseosos a trav\'es de la interfase gas-l\'iquido y posterior disoluci\'on (cuando es el caso).
 \item Transporte  de  masa  de  los  reactantes  a  trav\'es  de  la  capa  l\'imite soluci\'on-s\'olido, hacia la superficie del s\'olido.
 \item Reacci\'on qu\'imica o electroqu\'imica	en	la	superficie	del s\'olido, incluyendo  adsorci\'on  y  desorpci\'on  en  la  superficie  del  s\'olido  y/o  a trav\'es de la doble capa electroqu\'imica.
 \item Transporte de masa de las especies producidas a trav\'es de la capa l\'imite hacia el seno de la soluci\'on.
\end{enumerate}
Las reacciones homog\'eneas son generalmente m\'as r\'apidas que las reacciones heterog\'eneas, ya que necesitan transporte de masa en una sola fase y que las especies en soluci\'on reaccionan r\'apidamente. Por otra parte, las reacciones heterog\'eneas implican el transporte de masa a trav\'es del límite entre dos fases, lo que a veces es la etapa controlante de las reacciones. Las reacciones m\'as importantes en hidrometalurgía son heterog\'eneas, y a veces son controladas por el transporte de masa (difusi\'on).
En la figura siguiente se muestra un modelo simplificado de lixiviaci\'on, con formaci\'on de  una  capa  de  residuo  poroso.  Este  caso  es  tal  vez  el  m\'as  frecuente  en lixiviaci\'on. La reacci\'on de disoluci\'on de las especies sulfuradas de cobre con iones f\'erricos, corresponde a este caso. La lixiviaci\'on de minerales de baja ley, en las  que  el  material  est\'eril  o  ganga constituye  la  fracci\'on  mayoritaria,  pueden considerarse tambi\'en en ese grupo. La "capa" que se forma representa el est\'eril del mineral, mientras la disoluci\'on se propaga hacia el interior de la part\'icula.

\begin{figure}[H]
 \centering
 \includegraphics[scale=0.5]{Imagenes/Vectorial/capa.png}
 % capa.png: 648x563 pixel, 72dpi, 22.86x19.86 cm, bb=0 0 648 563
 \caption{Esquema de una reacci\'on de lixiviaci\'on con formaci\'on de una capa porosa}
 \label{fig:h}
\end{figure}

En este modelo, hay dos etapas adicionales:
\begin{itemize}
 \item Difusi\'on del reactivo a trav\'es de la capa s\'olida producida por la reacci\'on
(Producto poroso).
\item Transporte de masa de las especies solubles, productos de la reacci\'on, a trav\'es de la capa s\'olida formada, hacia el seno de la soluci\'on.
\end{itemize}

\paragraph{FACTORES QUE INFLUYEN EN LA CIN\'ETICA DE LIXIVIACI\'ON}
Cuando se modifica	la velocidad de reacci\'on de un proceso	 hidrometal\'urgico se debe tener en cuenta lo siguiente:
\begin{itemize}
 \item Tama\~no de part\'icula del mineral
 \item Concentraci\'on del reactante
\item Grado de agitaci\'on
\item Temperatura
\item Uso o presencia de catalizadores
\item Autocat\'alisis
\end{itemize}

\begin{itemize}
 \item[A]  Efecto del estado de divisi\'on del s\'olido
 \begin{itemize}
  \item Las reacciones intermoleculares tienen relaci\'on con los estados en que est\'an presentes.
  \item 	Los estados gaseosos y líquidos son los más \'optimos para lograr un contacto como reactantes.
\item En los s\'olidos cuantas m\'as peque\~nas son las part\'iculas que intervienen en la reacci\'on m\'as se incrementa su velocidad de reacci\'on.
 \item Al disminuir el tama\~no del s\'olido aumenta en relaci\'on cuadr\'atica la superficie disponible para la reacci\'on.
\item En una reacci\'on qu\'imica controlante puede dejar de serlo por le aumento de esta superficie \'util. A su vez la difusi\'on se ve favorecida al tener m\'as superficie de contacto o superficie reactiva disponible al disminuir el recorrido de los reactantes.
\item Esto se puede observar  en la oxidaci\'on superficial que es instant\'anea para los polvos muy fino de algunos metales, tal es el Cu obtenido por cementaci\'on	al ser expuestos en	oxidantes (atm\'osfera). Disminuyendo cuando	 se incrementa el tama\~no de part\'icula.
Existen casos en que al ir reaccionando un reactante (A) en el s\'olido (B), se va disolviendo una parte de este como (C) y en otros (D) como s\'olido.
\item De esta forma los reactantes (A) contin\'uan y deben atravesar difundiendo de la zona del nuevo s\'olido producido (D) hasta alcanzar la interfase de la reacci\'on en (B). Va avanzando hacia el interior del n\'ucleo sin reaccionar (B) como apreciamos en la siguiente figura.
\begin{figure}[H]
 \centering
 \includegraphics[scale=0.5]{Imagenes/Vectorial/sig.png}
 % sig.png: 876x506 pixel, 96dpi, 23.17x13.39 cm, bb=0 0 657 379
 \caption{Modelo de un n\'ucleo sin reaccionar.}
 \label{fig:s}
\end{figure}
 \end{itemize}
\item[B] Concentraci\'on del reactante:
\begin{itemize}
 \item 	Podemos mencionar la disoluci\'on de \'oxidos de Cu en \'acido sulf\'urico.
 \item Se observa que bajas concentraciones de \'acido el control es difusional, al aumentar el \'acido disponible se pasa a un control qu\'imico de acuerdo a la siguiente figura.
 \begin{figure}[H]
 \centering
 \includegraphics[scale=0.5]{Imagenes/Vectorial/conc.png}
 % conc.png: 785x451 pixel, 96dpi, 20.77x11.93 cm, bb=0 0 589 338
 \caption{Velocidad de disoluci\'on de \'oxidos de cobre e \'acido sulf\'urico}
 \label{fig:con}
\end{figure}
\end{itemize}
\item[C] Del grado de agitaci\'on:
\begin{itemize}
 \item En ciertas reacciones qu\'imicas al aumentar la agitaci\'on aumenta tambi\'en la velocidad de reacci\'on ''V''.
\item Entonces la difusi\'on controla el mecanismo global del proceso.
\item Se deduce que la velocidad global ''V'' aumentar\'a directamente con la velocidad de agitaci\'on RPM.
\item Ocurre porque disminuye el espesor de la capa límite y mejora la difusi\'on de los reactantes y/o de los productos.
\item V = (RPM)a
\item Se lograr\'a un valor del coeficiente exponente (a) de manera experimental para obtener datos y con un gr\'afico Log V versus Log (RPM) en el cual se encontrar\'a una recta con	pendiente tendr\'a el valor de ''a'' variando entre cero y uno.
\item Si a=0, la velocidad	 de reacci\'on	es independiente de la agitaci\'on y para a=1,	la velocidad de la reacci\'on	es directamente proporcional a la velocidad de agitaci\'on.
\item Si la velocidad global de la reacci\'on del proceso est\'a controlado por la velocidad de la reacci\'on qu\'imica; el proceso es independiente de la agitaci\'on y a = 0.
\item Un ejemplo es la disoluci\'on de Cu met\'alico en	soluci\'on amoniacal en presencia de $O_2$.
\begin{figure}[H]
 \centering
 \includegraphics[scale=0.5]{Imagenes/Vectorial/amon.png}
 % amon.png: 860x514 pixel, 96dpi, 22.75x13.60 cm, bb=0 0 645 385
 \caption{Velocidad de disoluci\'on del cobre en soluci\'on amoniacal}
 \label{fig:am}
\end{figure}
\end{itemize}
\item[D] La temperatura:
\begin{itemize}
 \item En la mayor\'ia de los casos el efecto de la aplicaci\'on de calor es el procedimiento m\'as efectivo para suministrar la energ\'ia de activaci\'on a las mol\'eculas reaccionantes.
\item Consiguientemente aumenta la velocidad de reacci\'on.
\item Adem\'as la velocidad de la reacci\'on qu\'imica se duplica con cada incremento de $10^oC$ en un sistema, siendo en algunos casos mayor.
\item Si las reacciones son endot\'ermicas, el suministro de calor sirve para que las mol\'eculas reaccionantes tengan energ\'ia de activaci\'on y promover la ocurrencia del proceso.
\item Si las reacciones son exot\'ermicas se producen dos efectos.\\
\emph{Primero se entrega	calor} para proporcionar la energ\'ia de activaci\'on, Sin la cual el proceso no se desarrolla o se hace lento.\\
Posteriormente cuando la reacci\'on desprenda calor pueda ser absorbido	por el	ambiente o incrementar la velocidad de reacci\'on, de ser as\'i termina en una explosi\'on o combusti\'on espont\'anea. Ejemplo la oxidaci\'on del carb\'on en polvo, con los productos pulvurulentos de Cu producidos en la cementaci\'on, concentrados sulfurados como la pirrotita.

\begin{figure}[H]
 \centering
 \includegraphics[scale=0.5]{Imagenes/Vectorial/temp.png}
 % temp.png: 741x625 pixel, 96dpi, 19.60x16.53 cm, bb=0 0 556 469
 \caption{Efecto de la temperatura sobre la solubilidad del oxigeno en agua}
 \label{fig:tem}
\end{figure}

\begin{figure}[H]
 \centering
 \includegraphics[scale=0.5]{Imagenes/Vectorial/pres.png}
 % pres.png: 754x659 pixel, 96dpi, 19.95x17.43 cm, bb=0 0 565 494
 \caption{Efecto de la temperatura y la presi\'on sobre la solubilidad del oxigeno en el agua}
 \label{fig:pr}
\end{figure}
\end{itemize}
\item[E] De catalizadores y autocat\'alisis:\\
Los catalizadores pueden retardar o acelerar la velocidad de una reacci\'on.
\begin{itemize}
 \item	Pudiendo suceder en el seno de	una fase l\'iquida o gaseosa; denomin\'andose cat\'alisis homog\'enea. Otra forma es la cat\'alisis heterog\'enea
 \item Las caracter\'isticas de un catalizador ser\'ian:
 \begin{enumerate}
  \item  Son sustancias l\'iquidas, s\'olidas o gaseosas.
\item La cantidad del catalizador que interviene no tiene relaci\'on con la cantidad del producto obtenido.
\item El catalizador persiste despu\'es de la reacci\'on.
 \item El catalizador no afecta la constante de equilibrio.
\item  El catalizador puede aumentar la velocidad de una de las reacciones secundarias que se produce de manera simult\'anea en el proceso y a la vez altera la principal.
\item Cada reacci\'on tiene su catalizador o catalizadores espec\'ificos.
 \item Se desconoce en parte como act\'ua un catalizador.
\item Puede ser	un efecto de	condensaci\'on en la	superficie del catalizador y que producir\'a un aumento en la concentraci\'on y por ende una mayor velocidad.
 \end{enumerate}
\begin{figure}[H]
 \centering
 \includegraphics[scale=0.5]{Imagenes/Vectorial/cata.png}
 % cata.png: 791x452 pixel, 96dpi, 20.93x11.96 cm, bb=0 0 593 339
 \caption{Velocidad de disoluci\'on de cobre con acido sulf\'urico diluido, mostrando el fen\'omeno de auto cat\'alisis}
 \label{fig:cay}
\end{figure}
\end{itemize}
\end{itemize}

\subsection{TIPOS DE LIXIVIACI\'ON}

Los m\'etodos disponibles para ejecutar la lixiviaci\'on de minerales, tratan de responder a las interrogantes fundamentales de toda actividad industrial, en t\'erminos de obtener el m\'aximo beneficio econ\'omico con el m\'inimo de costos.\\
Se trata entonces de lograr un balance econ\'omico entre los recursos aportados: Inversi\'on inicial, gastos operacionales (energ\'ia, reactivos, \'acido, agua, mano de obra, etc.) y los beneficios a obtener del procesamiento de las materias primas.\\

Diversos procesos unitarios son aplicados previos a la operaci\'on de lixiviaci\'on como:
\begin{enumerate}
 \item  Explotaci\'on minera y transporte de mineral.
\item  Chancado primario, secundario y terciario.
 \item  Molienda h\'umeda y clasificaci\'on.
 \item  Concentraci\'on, como la concentraci\'on gravitacional, la flotaci\'on, o una combinaci\'on de ambas.
 \item  Tratamiento qu\'imico previo, como en los concentrados refractarios, los cuales requieren de una tostaci\'on oxidante, reductora o clorurante.
\end{enumerate}



Siendo la lixiviaci\'on un proceso qu\'imico para acelerar su cin\'etica se le aplican los procedimientos desarrollados para mejorar el rendimiento cin\'etico, ellos se puede lograr aplicando:
\begin{enumerate}
 \item  Usando diferentes reactivos y lo variando su concentraci\'on.
 \item  Incorporando agitaci\'on.
 \item  Reducci\'on de tama\~no
 \item  Introduciendo el efecto de temperatura y presi\'on.
\end{enumerate}


El factor tiempo, es decir, la duraci\'on de los procesos seleccionados, es un elemento decisivo en la selecci\'on de un m\'etodo de lixiviaci\'on u otro, por su influencia sobre los costos, el tama\~no de los equipos y/o espacios involucrados.
El m\'etodo escogido para realizar la lixiviaci\'on depender\'a principalmente de un balance econ\'omico incluyendo valores de inversi\'on y de operaci\'on, que debe tomar en cuenta:
\begin{enumerate}
 \item El valor econ\'omico del metal, la ley de cabeza, el tonelaje disponible, el precio de venta y las condiciones de calidad impuestas por el mercado.
 \item  La recuperaci\'on que se puede esperar con cada m\'etodo.
 \item   El costo de mina, el m\'etodo de extracci\'on y de transporte del mineral a la planta.
 \item   El costo de los procesos de reducci\'on de tama\~no: chancado, molienda, clasificaci\'on y los eventuales pre tratamientos de aglomeraci\'on y/o curado.
 \item  El costo de los procesos de concentraci\'on y eventual pre-tratamiento t\'ermico, flotaci\'on, tuesta u otro proceso piro-metal\'urgico.
\end{enumerate}


\subsubsection{BIOLIXIVIACI\'ON ACIDA DE SULFUROS}

En general, las especies minerales sulfuradas son insolubles en agua, aun a altas temperaturas, sin embargo, desde el punto de vista de su comportamiento frente a la diluci\'on, los minerales sulfurados pueden clasificarse entre los que se disuelven:
\begin{enumerate}
 \item[a)] En presencia de ambientes reductores: Generando $H_2S$ si se trata de un medio acido, o bien liberando el ion sulfuro (S) si es ambiente alcalino. (Sulfuro de sodio, cianuro de sodio)

\item[b)] En presencia de Agentes Oxidantes : Generando Azufre elemental $S^o$ el que si bien en condiciones neutras y alcalinas se oxida a sulfato $(SO_4)$, en condiciones acidas puede mantenerse estable como tal. (I\'on f\'errico, cloro e hipoclorito, \'acido nítrico y nitratos, \'acido sulf\'urico concentrado, ox\'igeno)
\end{enumerate}

Reactivos qu\'imicos utilizados:
Bacterias del tipo bacillus oxidantes de azufre (Thiobacillus ferrooxidans, Thiobacillus thiooxidans, Thiobacillus thioparus, Sulfolobus, etc).
Bacterias reductoras de azufre (Desulfovibrio desulfuricans,Gallionella).
Bacterias oxidantes de fierro (Thiobacillus ferrooxidans,Sulfolobus,etc).
Hongos.
Algas microsc\'opicas.
Protozoos.
La lixiviaci\'on bacteriana de minerales es un fen\'omeno complejo al acoplar diversos elementos, algunos de los cuales son:
\begin{itemize}
 \item Actividad oxidativa, crecimiento, adherencia y transporte de microorganismos.
 \item Reacciones de disoluci\'on de minerales.
 \item Equilibrio i\'onico y transporte de especies y ox\'igeno entre la fase l\'iquida y el mineral.
 \item Reacciones de hidr\'olisis y precipitaci\'on de compuestos complejos en soluci\'on.
 \item Termoqu\'imica de las reacciones del sistema y transferencia de calor.
 \item Movimiento del aire y de la soluci\'on a trav\'es del lecho.
\end{itemize}

\begin{figure}[H]
 \centering
 \includegraphics[scale=0.35]{Imagenes/Vectorial/kech.png}
 % kech.png: 550x646 pixel, 72dpi, 19.40x22.79 cm, bb=0 0 550 646
 \caption{Diagrama de flujo para \'oxidos y sulfuros de baja ley hasta c\'atodos electro obtenidos}
 \label{fig:ke}
\end{figure}

Adem\'as de los factores que influyen en la lixiviaci\'on \'acida, las condiciones que afectan la cin\'etica de la lixiviaci\'on bacteriana son:\\

\paragraph{Aireaci\'on:} La acci\'on bacteriana, en cuanto a las reacciones de lixiviaci\'on de sulfuros, requiere de la presencia de una concentraci\'on m\'axima de ox\'igeno. Adem\'as por corresponder a un organismo aut\'otrofo, requiere di\'oxido de carbono como fuente de carbono para su metabolismo.\\

\paragraph{Nutrientes:} Para mantener la viabilidad de estos microorganismos, ellos necesitan energ\'ia y fuentes de elementos tales como: nitr\'ogeno, f\'osforo, magnesio, azufre, fierro, etc. \\

\paragraph{Temperatura:} El rango de temperaturas de crecimiento de estos microorganismos va desde 2 hasta $40^oC$, siendo el \'optimo del orden de $28\: a\: 35^oC$ dependiendo de la cepa bacteriana. 
pH: El rango de pH de crecimiento de estos microorganismos va desde 1,5 hasta 3,5, siendo el \'optimo del orden de 2,3.

\subsubsection{LIXIVIACI\'ON ACIDA DE \'OXIDOS}

Los principales minerales oxidados de cobre de inter\'es en Hidrometalurgia se presentan en la Tabla siguiente, donde tambi\'en figuran los sulfuros m\'as corrientes, todos ellos ordenados de acuerdo a la zona de oxidaci\'on a que pertenecen en el yacimiento, se puede apreciar que existen minerales oxidados de los que a priori puede se\~nalarse que son altamente solubles, como es el caso de los sulfatos y carbonatos de cobre. Adem\'as se muestra informaci\'on cualitativa de las cin\'eticas de disoluci\'on de las especies minerales de cobre .

\begin{equation}
 \begin{align}
 tenorita\: CuO+H_2SO_4 &\Leftrightarrow CuSO_4+H_2O \\
 cuprita\: Cu_2O+H_2SO_4 &\Leftrightarrow Cu_2SO_4+H_2O \\
 azurita\: Cu_3(0H)_2(CO_3)_2 + 3H_2SO_4 &\Leftrightarrow 3Cu_2SO_4+4H_2O +2CO_2 \\
 malaquita\:Cu_3(0H)_2(CO_3)_2 + 2H_2SO_4 &\Leftrightarrow 2Cu_2SO_4+3H_2O + CO_2
 \end{align}
\end{equation}

\begin{table}[H]
\label{tabla3}
\begin{center}
\begin{tabular}{|c|c|c|}
\hline
\multirow{2}{3cm}{CIN\'ETICA RELATIVA} & \multirow{2}{3cm}{TIEMPO DE REFERENCIA}
 &\multirow{2}{6cm}{ESPECIES MINERALES DE COBRE EN ESTA CATEGOR\'IA}\\
 & & \\
\hline 
\multirow{4}{3cm}{Muy r\'apida (temperatura ambiente)} & \multirow{4}{3cm}{Segundos a minutos, disoluci\'on es completa.}
 &\multirow{4}{6cm}{Carbonatos (malaquita, azurita), sulfatos (calcantita, brocantita) y cloruros (atacamita)}\\
 & & \\
 & & \\
 & & \\
\hline 
\multirow{2}{3cm}{R\'apida (requiere mayor acidez)} & \multirow{2}{3cm}{Horas, disoluci\'on es completa.}
 &\multirow{2}{6cm}{\'Oxidos cuprosos (tenorita) y silicatos (crisocola)}\\
 & & \\
\hline 
\multirow{5}{3cm}{Moderada (requieren oxidante)} & \multirow{5}{3cm}{D\'ias a semanas, disoluci\'on puede ser incompleta.}
 &\multirow{5}{6cm}{Cobre nativo, \'oxidos cuprosos (cuprita) y algunos silicatos y \'oxidos complejos con manganeso (neotocita, copper wad y copper pitch)}\\
 & & \\
 & & \\
 & & \\
 & & \\
\hline 
\multirow{3}{3cm}{Lenta (requieren oxidante)} & \multirow{3}{3cm}{Semanas a meses, disoluci\'on puede ser incompleta.}
 &\multirow{3}{6cm}{Sulfuros simples (digenita, calcosina y covelina)}\\
 & & \\
 & & \\
\hline 
\multirow{3}{3cm}{Muy lenta (requieren oxidante)} & \multirow{3}{3cm}{A\~nos, disoluci\'on es incompleta.}
 &\multirow{3}{6cm}{Sulfuros complejos (calcopirita, bornita, enargita y tetrahedrita)}\\
 & & \\
 & & \\
\hline 
\end{tabular}
\end{center}
\caption{Minerales De Cobre Y Su Disoluci\'on En Medio Acido}
\end{table}
  
Los m\'etodos m\'as caracter\'isticos son:
\begin{itemize}
 \item Lixiviaci\'on de lechos fijos
 \item Lixiviaci\'on in situ
 \item Lixiviaci\'on en bateas
 \item Lixiviaci\'on botaderos
 \item Lixiviaci\'on en pilas
 \item Lixiviaci\'on de pulpas
 \item Lixiviaci\'on en agitadores
 \item Lixiviaci\'on en autoclaves
\end{itemize}

\begin{figure}[H]
 \centering
 \includegraphics[scale=0.7]{Imagenes/Vectorial/df.png}
 % df.png: 725x566 pixel, 96dpi, 19.17x14.97 cm, bb=0 0 544 424
 \caption{Diagrama De Flujo Para Distintos Tipos De Mineral Y Su M\'etodo De Disoluci\'on.}
 \label{fig:ghh}
\end{figure}

\begin{table}[H]
\label{tabla3}
\begin{center}
\begin{tabular}{|c|c|c|c|c|}
\hline
 \multirow{3}{2.5cm}{Rangos de aplicaci\'on y resultados}&\multicolumn{4}{c|}{} \\
  & \multicolumn{4}{c|}{M\'etodos de Lixiviaci\'on}\\
  & \multicolumn{4}{c|}{}\\
\hline 
 &En botaderos&En pilas&Percolaci\'on&Agitaci\'on\\
\hline 
Ley del mineral&Baja ley&Baja-media&Media-alta&Alta ley\\
\hline
Tonelaje&Grande&Gran a mediano&Amplio rango&Amplio rango\\
\hline
Invesi\'on&Minima&Media&Media a alta&Alta\\
\hline 
\multirow{2}{2.5cm}{Granulometria}&\multirow{2}{2.3cm}{Corrido de mina}&\multirow{2}{2.3cm}{Chancado grueso}&\multirow{2}{2.3cm}{Chancado medio}&\multirow{2}{2.3cm}{Molienda h\'umeda}\\
 & & & & \\
\hline 
\multirow{2}{2.5cm}{Recuperaciones tipicas}&\multirow{2}{2.3cm}{40 a 50\%}&\multirow{2}{2.3cm}{50 a 70\%}&\multirow{2}{2.3cm}{70 a 80\%}&\multirow{2}{2.3cm}{80 a 90\%} \\
 & & & & \\
\hline
\multirow{2}{2.5cm}{Tiempo de tratamiento}&\multirow{2}{2.3cm}{Varios a\~nos}&\multirow{2}{2.3cm}{Varias semanas}&\multirow{2}{2.3cm}{Varios dias}&\multirow{2}{2.3cm}{Varias horas}\\
 & & & & \\
\hline
\multirow{3}{2.5cm}{Calidad de soluciones}&\multirow{3}{2.3cm}{Diluidas \\($1-2gpl\: Cu$)}&\multirow{3}{2.3cm}{Diluidas \\($1-6gpl\:Cu$)}&\multirow{3}{2.3cm}{Concentradas \\($20-40gpl\:Cu$)}&\multirow{3}{2.3cm}{Medianas \\($5-15gpl\:Cu$)} \\
 & & & & \\
 & & & & \\
\hline 
\multirow{14}{2.5cm}{Problemas principales en su aplicaci\'on}
&\multirow{14}{2.3cm}{
Recuperaci\'on incompleta.\\
Reprecipi- \\
taci\'on de Fe y Cu.\\
Canalizaciones.\\
Evaporaci\'on.\\
Perdidas de soluciones.\\
Soluciones muy diluidas.}
&\multirow{14}{2.3cm}{
Recuperaci\'on incompleta.\\
Requiere de grandes \'areas.\\
Canalizaciones.\\
Reprecipitaci\'on.\\
Evaporaci\'on.}
&\multirow{14}{2.3cm}{
Bloqueo por finos.\\
Requiere de mas inversi\'on.\\
Manejo de materiales.\\
Necesidad de mayor control en la planta.}
&\multirow{14}{2.3cm}{
Molienda.\\
Lavado en contracorriente.\\
Tanque de relaves.\\
Inversi\'on muy alta.\\
Control de la planta es mas sofisticado.}\\
 & & & & \\
 & & & & \\
 & & & & \\
 & & & & \\
 & & & & \\
 & & & & \\
 & & & & \\
 & & & & \\
 & & & & \\ 
 & & & & \\
 & & & & \\
 & & & & \\
 & & & & \\
\hline 
\end{tabular}
\end{center}
\caption{Resumen De Diferentes T\'ecnicas De Lixiviaci\'on De Minerales}
\end{table}
  
\paragraph{LIXIVIACI\'ON IN SITU-IN PLACE}
La lixiviaci\'on IN PLACE se refiere a la lixiviaci\'on de residuos fragmentados dejados en minas abandonadas.\\
La lixiviaci\'on IN SITU se refiere a la aplicaci\'on de soluciones directamente a un cuerpo mineralizado.\\
Dependiendo de la zona a lixiviar, que puede ser subterr\'anea o superficial, se distinguen tres tipos de lixiviaci\'on in situ:
\begin{itemize}
 \item Tipo I: Se trata de la lixiviaci\'on de cuerpos mineralizados fracturados situados cerca de la superficie, sobre el nivel de las aguas subterr\'aneas. Puede aplicarse a minas en desuso, en que se haya utilizado el \textacutedbl blockcaving\textgravedbl, o que se hayan fracturado hidr\'aulicamente o con explosivos (IN PLACE LEACHING).
 \item Tipo II: Son lixiviaciones IN SITU aplicadas a yacimientos situados a cierta profundidad bajo el nivel de aguas subterr\'anea, pero a menos de $300-500m$ de profundidad. Estos dep\'ositos se fracturan en el lugar y las soluciones se inyectan y se extraen por bombeo.
 \item Tipo III: Se aplica a dep\'ositos profundos, situados a m\'as de $500m$ bajo el nivel de aguas subterr\'aneas
\end{itemize}

\paragraph{LIXIVIACI\'ON EN BOTADEROS}
Esta t\'ecnica consiste en lixiviar lastres, desmontes o sobrecarga de minas de tajo abierto, los que debido a sus bajas leyes (por ej. $< 0.4\%Cu$) no pueden ser tratados por m\'etodos convencionales. Este material, generalmente al tama\~no \textacutedbl run of mine\textgravedbl es depositado sobre superficies poco permeables y las soluciones percolan a trav\'es del lecho por gravedad.\\ Normalmente, son de grandes dimensiones, se requiere de poca inversi\'on y es econ\'omico de operar, pero la recuperaci\'on es baja (por ej. $40-60\% Cu$) y necesita tiempos excesivos para extraer todo el metal.\\
Lixiviaci\'on en Botaderos: Es el tratamiento de minerales de bajas leyes, conocidos como \textacutedbl est\'eril mineralizado\textgravedbl y/o ripios de lixiviaci\'on.

\begin{figure}[H]
 \centering
 \includegraphics[scale=1]{Imagenes/Vectorial/bot.png}
 % bot.png: 279x234 pixel, 96dpi, 7.38x6.19 cm, bb=0 0 209 175
 \caption{Ejemplo de botadero}
 \label{fig:bot}
\end{figure}
  
Normalmente la lixiviaci\'on en botaderos es una operaci\'on de bajo rendimiento (pero tambi\'en de bajo costo). Entre las diferentes razones para ello se puede mencionar:
\begin{itemize}
 \item Gran tama\~no de algunas rocas ($> 1 m$).
 \item Baja penetraci\'on de aire al interior del botadero.
 \item Compactaci\'on de la superficie por empleo de maquinaria pesada.
 \item Baja permeabilidad del lecho y formaci\'on de precipitados.
 \item Excesiva canalizaci\'on de la soluci\'on favorecida por la heterogeneidad de tama\~nos del material en el botadero.
\end{itemize}

\paragraph{LIXIVIACI\'ON EN PILAS}
Se basa en la percolaci\'on de la soluci\'on lixiviante a trav\'es de un mineral chancado y apilado, el que est\'a formando una pila sobre un terreno previamente impermeabilizado. La pila se riega por aspersi\'on o goteo. Se aplica a minerales de alta ley debido a los costos de operaci\'on y transporte.\\
Existen dos tipos de pila seg\'un su operaci\'on.\\
\begin{itemize}
 \item Pila Permanente(capas m\'ultiples)
 \item Pila Renovable o Reutilizable  
\end{itemize}
\begin{figure}[H]
 \centering
 \includegraphics[scale=0.75]{Imagenes/Vectorial/lixb.png}
 % lixb.png: 608x449 pixel, 96dpi, 16.08x11.87 cm, bb=0 0 456 337
 \caption{Ejemplo de lixiviaci\'on en pila}
 \label{fig:libx}
\end{figure}

Configuraci\'on de la Pila\\
\begin{itemize}
 \item Pila Unitaria: todo el material depositado pasa por todas las etapas del ciclo de lixiviaci\'on, permitiendo una operaci\'on m\'as simple y flexible.
 \item Pila Din\'amica: coexisten materiales que est\'an en diversas etapas del ciclo de tratamiento.
\end{itemize}

\begin{figure}[H]
 \centering
 \includegraphics[scale=0.5]{Imagenes/Vectorial/conf.png}
 % conf.png: 721x584 pixel, 96dpi, 19.07x15.45 cm, bb=0 0 541 438
 \caption{Configuraci\'on normal delixiviaci\'on}
 \label{fig:cong}
\end{figure}

\begin{figure}[H]
 \centering
 \includegraphics[scale=0.5]{Imagenes/Vectorial/reci.png}
 % reci.png: 834x584 pixel, 96dpi, 22.08x15.46 cm, bb=0 0 626 438
 \caption{Reciclaje de soluci\'on lixiviante en contracorriente.}
 \label{fig:reci}
\end{figure}

El dise\~no de las pilas debe tener en cuenta los siguientes factores:
\begin{itemize}
 \item La calidad del patio o base de apoyo (impermeable)
 \item Las facilidades de riego y recolecci\'on o drenaje del efluente.
 \item La estabilidad de la pila seca y saturada en agua
 \item Los tanques (piscinas) de soluciones ricas y pobres
 \item La forma de apilamiento o deposici\'on del material lixiviable (Compactaci\'on, homogeneidad, etc.)
\end{itemize}

El sistema consiste en:
\begin{itemize}
 \item Una base firme y consolidada, debidamente preparada
 \item Una capa de lecho granular sobre el que apoyar suavemente la l\'amina
 \item La l\'amina o capa de impermeabilizaci\'on
 \item Un conjunto de drenaje o capa de recolecci\'on de l\'iquidos
 \item Una capa protectora del sistema
\end{itemize}

Generalmente, las membranas o l\'aminas de impermeabilizaci\'on del patio son geomembranas de origen sint\'etico (l\'aminas de pl\'astico:polietileno de alta densidad o PVC de 1 a 1.5mm o polietileno de baja densidad de 0.2 a 0.3mm de espesor) pero tambi\'en pueden ser materiales arcillosos compactados sobre el propio terreno, hormig\'on, asfalto, etc. Se pueden disponer de membranas o sellados simples, dobles o triples, de acuerdo con el n\'umero de capas impermeables o membranas que se hayan utilizado.

\paragraph{Riego de la pila}
El riego de las pilas se puede realizar fundamentalmente por dos procedimientos: por aspersi\'on o por distribuci\'on de goteo, este \'ultimo siendo recomendable en caso de escasez de l\'iquidos y bajas temperaturas. En la industria, se utiliza generalmente una tasa de riego del orden de $10-20\:\frac{litros}{h.m^2} $.El riego tiene que ser homog\'eneo.

\begin{figure}[H]
 \centering
 \includegraphics[scale=0.6]{Imagenes/Vectorial/ria.png}
 % ria.png: 778x391 pixel, 96dpi, 20.58x10.34 cm, bb=0 0 583 293
 \caption{Riego por aspersi\'on}
 \label{fig:ria}
\end{figure}

\begin{figure}[H]
 \centering
 \includegraphics[scale=0.75]{Imagenes/Vectorial/rig.png}
 % rig.png: 661x472 pixel, 96dpi, 17.48x12.48 cm, bb=0 0 495 354
 \caption{Riego por goteo}
 \label{fig:rig}
\end{figure}

Variables del proceso de lixiviaci\'on en pilas:
\begin{itemize}
 \item La granulometr\'ia
 \item La altura de la pila
 \item La tasa de riego [$\frac{l}{h.m^2}$] o [$\frac{l}{h.T}$]
 \item La concentraci\'on en \'acido de la soluci\'on de riego
 \item El tiempo de lixiviaci\'on
 \item Depende de la cin\'etica (lix. qu\'imica: 1 a 2 meses; lix. bacterial: 3 a 12 meses)
\end{itemize}

\paragraph{LIXIVIACI\'ON POR AGITACI\'ON}
La lixiviaci\'on por agitaci\'on se utiliza en los minerales de leyes m\'as altas, cuando los minerales generan un alto contenido de finos en la etapa de chancado, o cuando el mineral deseado está tan bien diseminado que es necesario molerlo para liberar sus valores y exponerlos a la soluci\'on lixiviante. Es tambi\'en el tipo de t\'ecnica que se emplea para lixiviar calcinas de tostaci\'on y concentrados.\\
La lixiviaci\'on en reactores, es solo aplicable a material finamente molido, ya sean lamas, relaves, concentrados o calcinas de tostaci\'on, y se realiza utilizando reactores agitados y aireados.\\
Esta operaci\'on permite tener un gran manejo y control del proceso de lixiviaci\'on. Adem\'as, la velocidad de extracci\'on del metal es mucho mayor que la lograda mediante el proceso de lixiviaci\'on en pilas o en bateas. Es un proceso de mayor costo, ya que incluye los costos de la molienda del mineral.\\
Sus ventajas comparativas con otros m\'etodos de lixiviaci\'on son:
\begin{itemize}
 \item Alta extracci\'on del elemento a recuperar
 \item Tiempos cortos de procesamiento (horas)
 \item Proceso continuo que permite una gran automatizaci\'on
 \item Facilidad para tratar menas alteradas o generadoras de finos
\end{itemize}
Sus desventajas son:
\begin{itemize}
 \item Un mayor costo de inversi\'on y operaci\'on
 \item Necesita una etapa de molienda y una etapa de separaci\'on s\'olido-l\'iquido (espesamiento y filtraci\'on).
\end{itemize}

\paragraph{VARIABLES DEL PROCESO DE LIXIVIACI\'ON POR AGITACI\'ON}

\paragraph{Granulometr\'ia:} 
El tama\~no de part\'iculas debe ser menor a 2 mm (problemas de embancamiento), pero no deben tener exceso de finos (menos de 40\% < 75 micrones) ya que dificultan la separaci\'on s\'olido-liquido.
\paragraph{Tiempo de agitaci\'on:} 
El tiempo necesario para una extracci\'on aceptable es muy importante para el proceso (velocidad de diluci\'on).
\paragraph{Mineralog\'ia del mineral:}
El tama\~no y la disposici\'on de la especie valiosa influye en el grado de molienda necesario para exponer esta especie a la soluci\'on de lixiviaci\'on.

\begin{figure}[H]
 \centering
 \includegraphics[scale=0.85]{Imagenes/Vectorial/agi.png}
 % agi.png: 445x328 pixel, 96dpi, 11.78x8.68 cm, bb=0 0 334 246
 \caption{M\'etodos de agitaci\'on}
 \label{fig:agi}
\end{figure}

\section{PURIFICACI\'ON DE SOLUCIONES: EXTRACCI\'ON POR SOLVENTES}
Los  procesos  de  purificaci\'on  y/o  concentraci\'on  se  pueden  dividir  en  varias categor\'ias:
\begin{itemize}
 \item Hidr\'olisis
 \item Cementaci\'on
 \item Precipitaci\'on de un compuesto espec\'ifico
 \item Extracci\'on por solventes
 \item Resinas de intercambio i\'onico
\end{itemize}

La Extracci\'on por Solventes es un proceso de separaci\'on que se emplea con tres fines fundamentales: concentrar, purificar y separar los elementos o metales disueltos en la soluci\'on rica generada en la etapa de la lixiviaci\'on.\\ 

Los objetivos del proceso de extracci\'on por solventes son:
\begin{itemize}
 \item Separaci\'on y purificaci\'on de \textacutedbl el\textgravedbl o \textacutedbl los\textgravedbl metales de inter\'es, desde las soluciones iniciales,  las  cuales contienen impurezas. En la separaci\'on se pueden extraer el o los metales de inter\'es o extraer las impurezas de la soluci\'on.
 \item Concentraci\'on de los metales disueltos, para disminuir los vol\'umenes a procesar.
 \item Transferencia de los metales disueltos, desde una soluci\'on acuosa compleja a otra soluci\'on acuosa diferente, que simplifique el proceso siguiente 
\end{itemize}

\begin{figure}[H]
 \centering
 \includegraphics[scale=0.75]{Imagenes/Vectorial/plix.png}
 % plix.png: 498x449 pixel, 96dpi, 13.17x11.88 cm, bb=0 0 373 337
 \caption{Planta Extracci\'on Por Solventes En Una Operaci\'on De Lixiviaci\'on De Cobre}
 \label{fig:plix}
\end{figure}

El proceso de extracci\'on por solventes (SX) define a un proceso de purificaci\'on y concentraci\'on de soluciones, al t\'ermino del cual, se generan soluciones aptas para su posterior tratamiento de precipitaci\'on electrol\'itica y comercializaci\'on directa del c\'atodo obtenido.

\paragraph{DESCRIPCI\'ON DEL PROCESO}
El proceso de extracci\'on por solventes se basa en una acci\'on reversible de intercambio i\'onico entre dos fases inmiscibles; la fase org\'anica (que contiene al extractante) y la fase acuosa: 
\begin{equation}
 [Cu^{++}]_A+2[HR]_O \Leftrightarrow [CuR_2]_O+[H^+]_A
\end{equation}
el sentido de la reacci\'on est\'a controlado por la acidez (pH) de la soluci\'on acuosa. En el proceso global de extracci\'on por solventes intervienen 2 etapas: de extracci\'on y de re-extracci\'on o stripping.
\begin{figure}[H]
 \centering
 \includegraphics[scale=0.45]{Imagenes/Vectorial/cic.png}
 % cic.png: 1078x532 pixel, 96dpi, 28.52x14.07 cm, bb=0 0 808 399
 \caption{Esquema C\'iclico (CONCEPTO De Anillos) Del Proceso Lixiviaci\'on-Extracci\'on por Solventes-Electro Obtenci\'on de Cu.}
 \label{fig:cic}
\end{figure}

\paragraph{FASE ORG\'ANICA}

\paragraph{Extractantes}
En la reacci\'on, el subíndice \textacutedbl o\textgravedbl define a la fase org\'anica y \textacutedbl A\textgravedbl a la fase acuosa. El reactivo org\'anico, propiamente tal, se representa por $HR$ y $CuR_2$ el complejo formado producto del intercambio, en la fase org\'anica. Extractante (tambi\'en llamado reactivo), es un compuesto que contiene un grupo funcional que es capaz de reaccionar qu\'imicamente con una especie particular de la fase acuosa. \\
Los extractantes se denominan $HR$ sin reaccionar, al extraer cobre se forma el complejo org\'anico $CuR_2$, disuelto en la fase org\'anica. Al formar el complejo met\'alico org\'anico $(CuR_2)$ se libera $H^+$ en el acuoso que constituir\'a el refino que vuelve a lixiviaci\'on; el $H^+$ (se puede hablar de recuperaci\'on de \'acido sulf\'urico) es el agente lixiviante de los minerales oxidados de cobre.\\ 

Para el tratamiento de las soluciones de lixiviaci\'on de minerales de cobre se emplean reactivos altamente selectivos para el $Cu^{2+}$, como los reactivos de la conocida serie LIX (LIX 63; LIX 64; LIX-64-N, LIX 70, etc.), los productos ACORGA (P-5100; P-5300) y otros, cuyo poder de extracci\'on para el $Cu^{2+}$ relativo a otros iones es muy fuerte.

\begin{table}[H]
\label{tabla5}
\begin{center}
\begin{tabular}{|c|c|c|c|}
\hline
\multirow{2}{1.7cm}{Reactivo} & \multicolumn{2}{c|}{Concentraci\'on $(H_2SO_4)gpl$}&Empleo\\
\cline{2-3} 
 &Extracci\'on&Stripping& \\
\hline 
\multirow{2}{1.7cm}{LIX 63}&\multirow{2}{*}{1}&\multirow{2}{*}{160}&\multirow{2}{5.5cm}{Extracci\'on de Cu de soluciones amoniacales (Ni, Co).}\\
 & & & \\
\hline 
\multirow{2}{1.7cm}{LIX 64}&\multirow{2}{*}{3}&\multirow{2}{*}{160-200}&\multirow{2}{5.5cm}{Extracci\'on de cobre de soluciones levemente \'acidas}\\
 & & & \\
\hline
\multirow{2}{1.7cm}{LIX 64 N}&\multirow{2}{*}{4-10}&\multirow{2}{*}{140-160}&\multirow{2}{5.5cm}{Idem anterior; pero selectivo para $Cu^{++}$ sobre i\'on f\'errico.}\\
 & & & \\
\hline
\multirow{3}{1.7cm}{LIX 70}&\multirow{3}{*}{30-40}&\multirow{3}{*}{200-400}&\multirow{3}{5.5cm}{En soluciones altamente \'acidas o con alta conc. de cobre y selectivo sobre $Fe^{+3}$}\\
 & & & \\
 & & & \\
\hline
\multirow{2}{1.7cm}{LIX 71(73)}&\multirow{2}{*}{10-15}&\multirow{2}{*}{220-230}&\multirow{2}{5.5cm}{Soluciones aciduladas de cobre y selectivo sobre $Fe^{+3}$}\\
 & & & \\
\hline
SME-529&4-10&150-160&Similar LIX 64 N\\
\hline
\multirow{3}{1.7cm}{ACORGA P-5100}&\multirow{3}{*}{3-10}&\multirow{3}{*}{170-175}&\multirow{3}{5.5cm}{En soluciones diluidas y/o concentrados en cobre y selectivo sobre $Fe^{+3}$}\\
 & & & \\
 & & & \\
\hline
\multirow{3}{1.7cm}{KELEX 100}&\multirow{3}{*}{>13}&\multirow{3}{*}{160-225}&\multirow{3}{5.5cm}{Extracci\'on de cobre de soluciones \'acidas y selectivo sobre $Fe^{+3}$}\\
 & & & \\
 & & & \\
\hline
\multirow{2}{1.7cm}{LIX 34}&\multirow{2}{*}{4-10}&\multirow{2}{*}{140-170}&\multirow{2}{5.5cm}{Para soluciones pobres en cobre, de alto hierro. Alta selectividad.}\\
 & & & \\
\hline
\end{tabular}
\end{center}
\caption{Condiciones de acidez en extractantes para cobre y empleo general}
\end{table}

Las propiedades que debe cumplir un extractante son:
\begin{itemize}
 \item[a)] Extraer el m\'aximo del elemento de inter\'es y minimizar la cantidad de extractante a usar.
 \item[b)] Elevada Capacidad de Saturaci\'on, la capacidad de saturaci\'on es la m\'axima concentraci\'on de especies valiosas que puede retener.
 \item[c)] Propiedades f\'isicas adecuadas para la transferencia de masa y separaci\'on de fases, tales como: densidad, viscosidad, etc.
 \item[d)] Selectividad, esta es una propiedad que mide la extracci\'on de determinadas especies en relaci\'on con la extracci\'on de otras.
 \item[e)] F\'acil reextracci\'on, para que un extractante sea adecuado metal\'urgicamente, debe existir un m\'etodo sencillo y barato para recuperar las especies extra\'idas.  
 \item[f)] Seguridad (bajo punto de inflamaci\'on, baja toxicidad, etc.,)
 \item[g)] Disponibilidad y costo, adem\'as de medios baratos de regeneraci\'on.
 \item[h)] Estabilidad qu\'imica bajo las condiciones de uso, un extractante debe ser relativamente estable, de tal modo que pueda ser usado  en muchos ciclos de extracci\'on.
\end{itemize}

\paragraph{Diluyentes}

El reactivo o agente de extracci\'on no se emplea normalmente solo, sino que suele estar mezclado en una cierta proporci\'on con un diluyente y en algunas ocasiones lleva un tercer componente o modificador para mejorar la separaci\'on de fases.  En varios trabajos se ha demostrado la importancia del diluyente como participante en el proceso de extracci\'on y no como un mero agente inerte como antes se pensaba.  El diluyente suele ser un hidrocarburo u otra substancia inmiscible con el agua y sus propiedades son:
\begin{itemize}
 \item[1] Debe ser capaz de disolver el reactivo de extracci\'on, tanto libre como en forma de complejo met\'alico.
 \item[2] Solubilidad en la fase acuosa muy peque\~na, para evitar o disminuir las p\'erdidas por disoluci\'on en la fase acuosa.
 \item[3] Mezclarse bien con el reactivo de extracci\'on para disminuir su viscosidad y facilitar el contacto entre fases.
 \item[4] Estabilidad qu\'imica en un amplio margen de condiciones de operaci\'on.
 \item[5] Punto de inflamaci\'on alto, para evitar riesgos de incendios.
 \item[6] No tener toxicidad para no presentar peligros durante la operaci\'on, y de bajo costo.
\end{itemize}

\section{ELECTRODEPOSITACI\'ON DEL COBRE}

\subsection{PRINCIPIOS}
\paragraph{Proceso de Electro-Obtenci\'on (EW):} 
La precipitaci\'on por reducci\'on electrol\'itica com\'unmente conocida como electro-obtenci\'on \'o electrodepositaci\'on (EW), es actualmente uno de los procedimientos m\'as sencillos para recuperar metales en forma pura y selectiva, respecto de las impurezas existentes en la soluci\'on, su característica principal es que el metal ya est\'a en la soluci\'on y solamente se trata de recuperarlo deposit\'andolo en el c\'atodo, mientras el \'anodo es esencialmente insoluble.

\paragraph{a)  Proceso de Electrodeposici\'on (EW) :}
En este proceso el metal viene disuelto y el \'anodo es esencialmente insoluble

\paragraph{b)  Procesos de Electro Refinaci\'on (ER):}
El \'anodo contiene el metal que se disuelve, a la vez que se produce el dep\'osito en el c\'atodo.

\paragraph{c)  Galvanoplastia (Electroplating):}
Consiste en recubrimientos met\'alicos delgados con fines anticorrosivos y est\'eticos (cromados).

\paragraph{d)  Electro conformado  (Electroforming):}
Es la elaboraci\'on de piezas met\'alicas especiales por v\'ia electrol\'itica.

\paragraph{Celdas de electr\'olisis}

Los procesos electrometal\'urgicos tienen lugar en unidades llamadas CELDAS DE ELECTROLISIS, las cuales se agrupan para constituir la nave o planta electrol\'itica. Una celda de electr\'olisis est\'a constituida por:

\begin{itemize}
 \item La celda: \\
 Es un recipiente que contiene el electrolito y los electrodos. En algunos casos, la celda puede ser constituida por dos mitades, conectadas entre s\'i por un puente salino.
 \item El electrolito: \\
Un medio acuoso, que contiene los iones del metal a depositar y    otros iones que migran permitiendo el paso de la corriente entre los electrodos.
 \item El \'anodo: \\
Material s\'olido conductor en cuya superficie se realiza un proceso de oxidaci\'on con liberaci\'on de electrones. 
\begin{equation}
 \text{Ejemplo:} Pb \Rightarrow Pb^{2+} + 2e^-
\end{equation}
 \item El c\'atodo:\\ 
Electrodo s\'olido conductor en cuya superficie se realiza un proceso de reducci\'on con los electrones provenientes del \'anodo.
 \begin{equation}
 \text{Ejemplo:}   Cu^{2+}+ 2e^- \Rightarrow Cu
 \end{equation}
\begin{figure}[H]
 \centering
 \includegraphics[scale=0.5]{Imagenes/Vectorial/cel.png}
 % cel.png: 607x391 pixel, 72dpi, 21.41x13.79 cm, bb=0 0 607 391
 \caption{Celda electrol\'itica}
 \label{fig:cel}
\end{figure}

 \begin{figure}[H]
 \centering
 \includegraphics[scale=0.5]{Imagenes/Vectorial/ceg.png}
 % ceg.png: 624x389 pixel, 72dpi, 22.01x13.72 cm, bb=0 0 624 389
 \caption{Celda galv\'anica}
 \label{fig:ceg}
\end{figure}
\end{itemize}

\subsection{LEY DE FARADAY}

La ley de Faraday establece que la masa de metal depositado es proporcional a la cantidad de corriente que circula a trav\'es de la celda y al tiempo de operaci\'on de la electr\'olisis.\\
Se expresa as\'i:
\begin{equation}
 m_F=\frac{M}{nF}Ixt
\end{equation}

Donde:\\

\begin{tabular}{cc}
mF:&masa depositada (g) \\
M:&Peso molecular del metal depositado \\
n:&Valencia del ion met\'alico en la soluci\'on \\
F:&\multirow{2}{7cm}{Constante de Faraday ($96487\frac{Coulomb}{equivalente}$) \\
(1 coulomb = 1 A x 1 s)} \\
 & \\
I:&Corriente que circula (A) \\
t:&Tiempo de operaci\'on de la electr\'olisis (seg)\\
\end{tabular}\\


El equivalente electroqu\'imico (EEQ) de la sustancia transformada, se define como la cantidad de sustancia que es afectada por el paso de una corriente de 1 A en 1 hora.\\
El equivalente electroqu\'imico de un metal queda determinado por las siguientes constantes:

\begin{equation}
 EEQ=\frac{M}{nF}x3600
\end{equation}
Para el caso del cobre ($EEQ =1.18\frac{Kg}{kA.h}$), la expresi\'on anterior queda:
\begin{equation}
 m_F=1.18 Ixt
\end{equation}

Donde:\\

\begin{tabular}{cc}
$m_F$:&masa de cobre depositada (kg)\\
I:&Corriente que circula (kA)\\
t:&Tiempo de operaci\'on de la electr\'olisis (h)\\
\end{tabular}

\paragraph{Eficiencia de corriente:}

La masa que se obtiene con la ecuaci\'on de Faraday ($m_F$) es te\'orica o estequiom\'etrica, ya que considera que toda la corriente que circula se aprovecha solo para depositar el metal, pero en los procesos reales de EW de cobre, hay reacciones par\'asitas y la masa depositada realmente ($m_R$) es menor a lo que se esperaba.\\
La EFICIENCIA DE CORRIENTE se define como la raz\'on entre la cantidad de cobre depositada y la que se deber\'ia haber depositado te\'oricamente seg\'un la ley de Faraday:\\
\begin{equation}
 EC=\frac{m_R}{m_F}
\end{equation}
Como concepto, indica la fracci\'on de corriente que es efectivamente utilizada en la reacci\'on de depositaci\'on de cobre. As\'i, por ejemplo, si la EC es de 0.8 (80\%), significa que solamente el 80\% de la corriente est\'a siendo \'utilmente utilizado y el 20\% restante est\'a siendo empleado en reacciones paralelas o parasitarias, fugas, etc.\\
Por ejemplo, en EW de Cu, 

\begin{equation}
 \begin{align}
  \text{Reacci\'on principal}&Cu^{2+} + 2e^- \Rightarrow Cu \:\:97\% I \\
  \text{Reacci\'on par\'asita}&Fe^{3+} + 1e^- \Rightarrow Fe^{2+}\: 3\% I
 \end{align}
\end{equation}

\subsection{ESPECIFICACIONES DE CALIDAD DE LOS C\'ATODOS}
La EW, como etapa final del proceso hidrometal\'urgico, tiene entre sus objetivos producir c\'atodos de cobre de alta pureza para maximizar los resultados econ\'omicos de venta del producto. Los procesos LX/SX/EW han logrado un desarrollo y potencialidad para producir cobre de alta pureza con una calidad superior o similar al cobre electrorefinado. El c\'atodo \textacutedbl grado A\textgravedbl contiene MAS DE 99.96\% de Cu .\\

En general, los c\'atodos electro obtenidos producidos por medio de SX/EW presentan bajos niveles de impurezas de baja tolerancia, como son los elementos: ars\'enico (As), selenio (Se), bismuto (Bi) y antimonio (Sb). Las impurezas que m\'as problemas presentan son el plomo (Pb), azufre (S) y fierro (Fe). Los niveles de estas \'ultimas impurezas en los c\'atodos, est\'an influenciadas significativamente por la pr\'actica operacional empleada en las plantas de electro obtenci\'on.\\

La calidad qu\'imica de los c\'atodos est\'a ligada fuertemente a la calidad f\'isica o apariencia presentada por el dep\'osito; estableci\'endose en la pr\'actica operacional, que un deposito liso, denso y coherente, presenta mejor calidad qu\'imica que otro rugoso, poroso e incoherente. Eso se debe a que, en el primer caso, la soluci\'on que contiene iones sulfatos y part\'iculas de plomo, no queda trampeada en posibles huecos del dep\'osito. \\

De acuerdo al nivel de impurezas, los c\'atodos pueden ser clasificados en:
\begin{itemize}
 \item C\'atodos de baja calidad ( Low Grade )
 \item C\'atodos est\'andar ( Standard Grade )
 \item C\'atodos de alta calidad no registrados ( High Grade, Unregistered )
 \item C\'atodos de alta calidad registrados ( High Grade registrados en LME o COMEX)
\end{itemize}

A los c\'atodos High Grade se les clasifica tambi\'en como \textacutedbl C\'atodos Grado A\textgravedbl ( LME ) en base a la norma y la norma ASTM B-115/91( Comex ) y como Cu-Cath 1 seg\'un la norma BS 6017:1981 \textacutedbl Specification for Copper Refinery Shapes.\textgravedbl


\begin{table}[H]
\label{tabla6}
\begin{center}
\begin{tabular}{|c|c|c|c|c|c|}
\hline
\multirow{4}{1.5cm}{Grupo de elementos}&\multirow{4}{1.5cm}{Elemento}&\multirow{4}{1.6cm}{Concent. m\'axima BS 6017}&\multirow{4}{1.6cm}{Concent. m\'axima
ASTM  B}&\multirow{4}{2cm}{Concent. m\'axima del grupo BS 6017}&\multirow{4}{2cm}{Concent. m\'axima del grupo ASTM B}\\
 & & & & & \\
 & & & & & \\
 & & & & & \\ 
\hline
\multirow{3}{1.5cm}{\:\:\:\:\:\:\:\:1}&Se&2&4& & \\
\cline{2-6}
 &Te&2&2&3&5\\
\cline{2-6} 
 &Bi&2&2& & \\
\hline
\multirow{6}{1.5cm}{\:\:\:\:\:\:\:\:2}&Cr& & & & \\
\cline{2-6}
 &Mn& & & & \\
\cline{2-6} 
 &Sb&4&5&15& \\
 \cline{2-6}
 &Cd& & & & \\
\cline{2-6} 
 &As&5&5& & \\
 \cline{2-6}
 &P& & & & \\
\hline
3&Pb&5&8&5&8\\
\hline
4&S&15*&15&15&15\\
\hline
\multirow{6}{1.5cm}{\:\:\:\:\:\:\:\:5}&Sn& &10& & \\
\cline{2-6}
 &Ni& &8& & \\
\cline{2-6} 
 &Fe&10&12& &15 \\
 \cline{2-6}
 &Si& & & & \\
\cline{2-6} 
 &Zn& & & & \\
 \cline{2-6}
 &Co& & & & \\
\hline
6&Ag&25&25&25&25\\
\hline
7&O&**&20& & \\
\hline
 &Total&65&90& & \\
\hline
\end{tabular}
\end{center}
\caption{Concentraciones m\'aximas de impurezas en los c\'atodos de cobre electrolitico (ppm).}
\end{table}

\subsection{MUESTREO Y AN\'ALISIS DE IMPUREZAS}
El estado actual del arte en lo que se refiere a la determinaci\'on de impurezas  de los c\'atodos electro obtenidos, tiene relaci\'on con los siguientes aspectos fundamentales:
\begin{itemize}
 \item Procedimientos de selecci\'on,  tratamiento y toma de muestras.
 \item Procedimientos de an\'alisis de las impurezas contenidas.
 \item Procedimientos de selecci\'on,  tratamiento y manipulaci\'on.
\end{itemize}

Pr\'acticas normales de los procedimientos de selecci\'on, recolecci\'on, lavado, se presentan en las tablas 2.7, 2.8, 2.9, 2.10.

\begin{table}[H]
\label{tabla7}
\begin{center}
\begin{tabular}{|c|c|c|}
\hline
\multirow{4}{2cm}{Lavado previo de C\'atodos}&\multirow{4}{5cm}{Lavado en la l\'inea normal de proceso de lavado\\ Lavado separado de c\'atodos de muestra}&\multirow{4}{5cm}{Puede presentar desviaciones sistem\'aticas de calidad por no eliminaci\'on de sales o borras.}\\
 & &  \\
 & &  \\
 & &  \\ 
\hline
\multirow{5}{2cm}{Inspecci\'on f\'isica}&\multirow{5}{5cm}{Verificaci\'on de existencia de n\'odulos o manchas y registro de calidad f\'isica}&\multirow{5}{5cm}{Se debe disponer de un manual de inspecci\'on y catalogaci\'on de defectos t\'ipicos que ameriten el rechazo.}\\
 & &  \\
 & &  \\
 & &  \\ 
 & & \\
\hline
\multirow{3}{2cm}{Limpieza de c\'atodos}&\multirow{3}{5cm}{Eliminaci\'on de n\'odulos aislados o rechazo seg\'un procedimiento}&\multirow{3}{5cm}{Operaci\'on manual.}\\
 & &  \\
 & &  \\
\hline 
\multirow{6}{2cm}{Pesaje de C\'atodos}&\multirow{6}{5cm}{Pesaje en l\'inea \\
Pesaje en b\'asculas certificadas del total de los c\'atodos \\ Verificaciones y contrastaciones de b\'asculas.}&\multirow{6}{5cm}{Es importante el control metrol\'ogico de las b\'asculas}\\
 & &  \\
 & &  \\
 & &  \\ 
 & & \\
 & & \\
\hline
\multirow{4}{2cm}{Manipula-\\ ci\'on y Transporte}&\multirow{4}{5cm}{Control de enzunchado, carga, descarga y control de contaminaciones, da\~nos o deformaciones}&\multirow{4}{5cm}{Es importante las condiciones del ambiente de manejo de c\'atodos.}\\
 & &  \\
 & &  \\
 & &  \\ 
\hline
\multirow{6}{2cm}{Lugar de muestreo}&\multirow{6}{5cm}{Toma de muestras en sala acondicionada ( Taladrado, punzando u otro ) \\ Toma de muestras en forma contigua a l\'inea de producci\'on.}&\multirow{6}{5cm}{Operaciones de muestreo pueden ser automatizadas \\ Debe ser analizada}\\
 & &  \\
 & &  \\
 & &  \\ 
 & & \\
 & & \\
\hline
\end{tabular}
\end{center}
\caption{Procedimientos de selecci\'on, recolecci\'on, lavado de c\'atodos.}
\end{table}

\begin{center}
\begin{tabular}{|c|c|c|c|}
\hline
Metodo&Ventajas&Desventajas&Observaciones\\
\hline 
\multirow{30}{1.2cm}{\rotatebox{90}{Taladrado}}&\multirow{30}{3.5cm}{Baja inversi\'on. \\Rapidez \\Automatizable \\Si el proceso productivo est\'a bajo control y la distribuci\'on de impurezas es homog\'enea, el taladrado es suficiente para obtener muestras para la determinaci\'on de impurezas}&\multirow{30}{3cm}{Alto costo de operaci\'on si no es automatizado.\\ Mala representatividad \\ Tama\~no de viruta inadecuado para an\'alisis directo.\\ Riesgo de contaminaci\'on con fierro. \\ Riesgo de alteraci\'on de oxidaci\'on y p\'erdida de contaminantes por calentamiento con la broca.\\
No puede ser utilizado en la l\'inea de producci\'on y los c\'atodos deben ser separados para la obtenci\'on de la muestra.}&\multirow{30}{3.5cm}{Para an\'alisis qu\'imico elemental por absorci\'on at\'omica u otras t\'ecnicas, requiere de viruta obtenida del c\'atodo o de lingote obtenido despu\'es de fusi\'on en atm\'osfera controlada.\\ Para an\'alisis espectrom\'etricos se requiere de fusi\'on en atm\'osfera controlada  y granallado o bien de peletizado de la viruta obtenida del lingote.\\Las normas no se pronuncian sobre este m\'etodo y no es recomendado como procedimiento en caso de disputas comerciales.\\ Plantilla de muestreo debe ser definida por distribuci\'on de impurezas.}\\
 & & & \\
 & & & \\
 & & & \\ 
 & & & \\
 & & & \\
 & & & \\  
 & & & \\
 & & & \\
 & & & \\ 
 & & & \\
 & & & \\
 & & & \\
 & & & \\ 
 & & & \\
 & & & \\
 & & & \\  
 & & & \\
 & & & \\
 & & & \\ 
 & & & \\
 & & & \\
 & & & \\ 
 & & & \\
 & & & \\
 & & & \\ 
 & & & \\
 & & & \\
 & & & \\  
 & & & \\
\hline
\end{tabular}
\end{center}


\begin{center}
\begin{tabular}{|c|c|c|c|}
\hline
Metodo&Ventajas&Desventajas&Observaciones\\
\hline
\multirow{25}{1.2cm}{\rotatebox{90}{Punzonado}}&\multirow{25}{3.5cm}{Inversi\'on moderada \\Rapidez\\Automatizable\\Puede ser incluido en la l\'inea de proceso para la obtenci\'on de los discos.\\ Si el proceso productivo esta bajo control y la distribuci\'on de impurezas es homog\'enea, el punzonado es adecuado como procedimiento de obtenci\'on de muestras para la determinaci\'on de impurezas.}&\multirow{25}{3cm}{Alto costo de operaci\'on si no es automatizado.\\ Mala representatividad. \\ Los discos deben ser fundidos en atm\'osfera controlada.\\ Riesgo de contaminaci\'on y oxidaci\'on en la manipulaci\'on para obtenci\'on de la muestra.}&\multirow{25}{3.5cm}{Para an\'alisis qu\'imico elemental por absorci\'on at\'omica u otras t\'ecnicas, requiere de viruta obtenida despu\'es de la fusi\'on en atm\'osfera controlada de los discos. \\ Para an\'alisis espectrom\'etricos se requiere de fusi\'on y granallado o bien an\'alisis en equipo de chispa sobre los discos
Propuesto para evaluaci\'on en BSI DD96:1984.\\ Plantilla de muestreo debe ser definida por distribuci\'on de impurezas}\\
 & & & \\  
 & & & \\
 & & & \\
 & & & \\ 
 & & & \\
 & & & \\
 & & & \\
 & & & \\ 
 & & & \\
 & & & \\
 & & & \\  
 & & & \\
 & & & \\
 & & & \\ 
 & & & \\
 & & & \\
 & & & \\ 
 & & & \\
 & & & \\
 & & & \\ 
 & & & \\
 & & & \\
 & & & \\  
 & & & \\
\hline
\end{tabular}
\end{center}


\begin{center}
\begin{tabular}{|c|c|c|c|}
\hline
Metodo&Ventajas&Desventajas&Observaciones\\
\hline
\multirow{23}{1.2cm}{\rotatebox{90}{Aserrado}}&\multirow{23}{3.5cm}{Buena representatividad \\ Inversi\'on moderada \\ Automatizable \\ Muy lento, 10 a 15 minutos por c\'atodo}&\multirow{23}{3cm}{Alto costo de operaci\'on si no es automatizado. \\ Riesgo de alteraci\'on de oxidaci\'on y p\'erdida de contaminantes por calentamiento con la sierra.}&\multirow{23}{3.5cm}{Propuesto para evaluaci\'on en BSI DD96:1984.\\ Para an\'alisis espectrom\'etricos se requiere de fusi\'on en atm\'osfera controlada y granallado o bien de peletizado de la viruta obtenida de lingote despu\'es de fusi\'on \\
Reconocido por la literatura como un procedimiento de buena representatividad alternativo al seccionado. \\No requiere de plantilla de muestreo} \\
 & & & \\
 & & & \\ 
 & & & \\
 & & & \\
 & & & \\
 & & & \\ 
 & & & \\
 & & & \\
 & & & \\  
 & & & \\
 & & & \\
 & & & \\ 
 & & & \\
 & & & \\
 & & & \\ 
 & & & \\
 & & & \\
 & & & \\ 
 & & & \\
 & & & \\
 & & & \\  
 & & & \\
\hline
\end{tabular}
\end{center}


\begin{table}[H]
\label{tabla8}
\begin{center}
\begin{tabular}{|c|c|c|c|}
\hline
Metodo&Ventajas&Desventajas&Observaciones\\
\hline
\multirow{20}{1.2cm}{\rotatebox{90}{Seccionado}}&\multirow{20}{3.5cm}{Buena representatividad \\Aceptado por normas}&\multirow{20}{3cm}{Alta inversi\'on\\
Costo de operaci\'on muy alto \\ Procedimiento lento \\No automatizable}&\multirow{20}{3.5cm}{Recomendado como m\'etodo arbitral por la norma ASTM B115 - 91.\\ Para an\'alisis espectrom\'etricos se requiere de fusi\'on en atm\'osfera controlada y granallado o bien de peletizado de la viruta obtenida del lingote producto de la fusi\'on.\\ Propuesto para evaluaci\'on en BSI DD96:1984. \\ No requiere de plantilla de muestreo}\\
 & & & \\
 & & & \\
 & & & \\ 
 & & & \\
 & & & \\
 & & & \\  
 & & & \\
 & & & \\
 & & & \\ 
 & & & \\
 & & & \\
 & & & \\ 
 & & & \\
 & & & \\
 & & & \\ 
 & & & \\
 & & & \\
 & & & \\  
 & & & \\
\hline
\end{tabular}
\end{center}
\caption{Procedimientos de selecci\'on, recolecci\'on, lavado de c\'atodos.}
\end{table}


\begin{center}
\begin{tabular}{|c|c|c|c|}
\hline
\multirow{3}{1.5cm}{Grupo de elementos}&\multirow{3}{1.5cm}{Elemento}&\multirow{3}{3cm}{M\'etodo}&\multirow{3}{6cm}{Norma}\\
 & & & \\
 & & & \\
\hline
\multirow{6}{1.5cm}{\:\:\:\:\:\:\:\:1}&\multirow{2}{1.5cm}{Se}&\multirow{2}{3cm}{AA Hidruros \\ AA Horno grafito} &\multirow{2}{6cm}{BS 7317-3 1990 \\
BS 7317-4 1990 y ASTM B 115-91} \\
 & & & \\
\cline{2-4}
 &\multirow{2}{1.5cm}{Te}&\multirow{2}{3cm}{AA Hidruros \\ AA Horno grafito} &\multirow{2}{6cm}{BS 7317-3 1990 \\
BS 7317-4 1990 y ASTM B 115-91} \\
 & & & \\
\cline{2-4}
 &\multirow{2}{1.5cm}{Bi}&\multirow{2}{3cm}{AA Hidruros \\ AA Horno grafito} &\multirow{2}{6cm}{BS 7317-3 1990 \\
BS 7317-4 1990 y ASTM B 115-91} \\
 & & & \\
\cline{1-4}
\multirow{11}{1.5cm}{\:\:\:\:\:\:\:\:2}&\multirow{2}{1.5cm}{Cr}&\multirow{2}{3cm}{AA Vol\'umenes discretos} &\multirow{2}{6cm}{BS 7317-2 1990} \\
 & & & \\
 \cline{2-4}
 &\multirow{2}{1.5cm}{Mn}&\multirow{2}{3cm}{AA Vol\'umenes discretos} &\multirow{2}{6cm}{BS 7317-1 1990} \\
 & & & \\
\cline{2-4}
&\multirow{2}{1.5cm}{Sb}&\multirow{2}{3cm}{AA Hidruros \\ AA Horno grafito} &\multirow{2}{6cm}{BS 7317-3 1990 \\
BS 7317-4 1990 y ASTM B 115-91} \\
 & & & \\
\cline{2-4}
&\multirow{2}{1.5cm}{Cd}&\multirow{2}{3cm}{AA Vol\'umenes discretos} &\multirow{2}{6cm}{BS 7317-1 1990} \\
 & & & \\
\cline{2-4}
&\multirow{2}{1.5cm}{As}&\multirow{2}{3cm}{AA Hidruros \\ AA Horno grafito} &\multirow{2}{6cm}{BS 7317-3 1990 \\
BS 7317-4 1990 y ASTM B 115-91} \\
 & & & \\
\cline{2-4}
 &P&Colorim\'etrico&BS 7317-6 1990\\
\hline 
\multirow{3}{1.5cm}{\:\:\:\:\:\:\:\:3}&\multirow{3}{1.5cm}{Pb}&\multirow{3}{3cm}{AA Horno grafito \\ AA colecci\'on con lantano} &\multirow{3}{6cm}{BS 7317-4 1990 y ASTM B 115-91 \\ BS 7317-7 1990} \\
 & & & \\
 & & & \\
\cline{1-4}
\multirow{2}{1.5cm}{\:\:\:\:\:\:\:\:4}&\multirow{2}{1.5cm}{S}&\multirow{2}{3cm}{Oxidaci\'on (Leco)\\Generaci\'on $H_2S$} &\multirow{2}{6cm}{BS 7317-5 1990} \\
 & & & \\
\hline
\multirow{13}{1.5cm}{\:\:\:\:\:\:\:\:5}&\multirow{2}{1.5cm}{Sn}&\multirow{2}{3cm}{AA Hidruros \\AA Horno grafito} &\multirow{2}{6cm}{BS 7317-3 1990 \\
BS 7317-4 1990 y ASTM B 115-91} \\
 & & & \\
\cline{2-4}
 &\multirow{3}{1.5cm}{Ni}&\multirow{3}{3cm}{AA Vol\'umenes discretos \\AA Horno grafito} &\multirow{3}{6cm}{BS 7317-2 1990 \\ ASTM B 115-91} \\
 & & & \\
 & & & \\
\cline{2-4}
&\multirow{3}{1.5cm}{Fe}&\multirow{3}{3cm}{AA Vol\'umenes discretos \\AA Horno grafito} &\multirow{3}{6cm}{BS 7317-2 1990 \\ ASTM B 115-91} \\
 & & & \\
 & & & \\
\cline{2-4}
  &Si&Colorim\'etrico&BS 7317-6 1990\\
\cline{2-4}  
&\multirow{2}{1.5cm}{Zn}&\multirow{2}{3cm}{AA Vol\'umenes discretos} &\multirow{2}{6cm}{BS 7317-2 1990} \\
 & & & \\
\cline{2-4}
&\multirow{2}{1.5cm}{Co}&\multirow{2}{3cm}{AA Vol\'umenes discretos} &\multirow{2}{6cm}{BS 7317-2 1990} \\
 & & & \\
\cline{1-4}
\end{tabular}
\end{center}


\begin{table}[H]
\label{tabla9}
\begin{center}
\begin{tabular}{|c|c|c|c|}
\hline
\multirow{2}{1cm}{6}&\multirow{2}{1cm}{Ag}&\multirow{2}{4cm}{AA \\AA Horno grafito}&\multirow{2}{4cm}{BS 7317-1 1990 \\ ASTM B 115-91} \\
 & & & \\
\cline{1-4}
\multirow{2}{1cm}{7}&\multirow{2}{1cm}{O}&\multirow{2}{4cm}{Combusti\'on detector \\ infrarrojo (Leco )}&\multirow{2}{4cm}{ASTM B 115-91} \\
 & & & \\
\cline{1-4}
\multirow{2}{1cm}{}&\multirow{2}{1cm}{Cl}&\multirow{2}{4cm}{Indirecto \\ (Gravimetr\'ia, AA)}&\multirow{2}{4cm}{No normalizado} \\
 & & & \\
\cline{1-4}
\multirow{2}{1cm}{}&\multirow{2}{1cm}{Cu}&\multirow{2}{4cm}{Electrogravimetr\'ia}&\multirow{2}{4cm}{ISO 1553-1976(E) \\ASTM E-53-86a} \\
 & & & \\
\cline{1-4}
\end{tabular}
\end{center}
\caption{Procedimientos de an\'alisis qu\'imico de las impurezas contenidas}
\end{table}

\begin{table}[H]
\label{tabla10}
\begin{center}
\begin{tabular}{|c|c|c|c|}
\hline
M\'etodo&Elemento&Ventajas&Desventajas\\
\hline
\multirow{6}{2.2cm}{Espectro- \\metr\'ia de emisi\'on}&\multirow{6}{2cm}{As, Sb, Fe, Ni, Pb, Bi, Te, Zn, Sn, Cd, Ag}&\multirow{6}{2.5cm}{Rapidez de an\'alisis}&\multirow{6}{5.2cm}{Preparaci\'on de muestras lenta.
Alta inversi\'on. Requiere de buenos est\'andares. L\'imite de detecci\'on limitado Precisi\'on limitada. Rango de calibraci\'on limitado} \\
 & & & \\
 & & & \\
 & & & \\
 & & & \\
 & & & \\
\hline
\multirow{4}{2.2cm}{Absorci\'on at\'omica simple(DVN) Especial}&\multirow{4}{2cm}{Ag, Ni, Cd, Mn, Co, Fe, Cr,  Zn, Pb, Cl, Au}&\multirow{4}{2.5cm}{Inversi\'on baja. Amplio rango de calibraci\'on. Vers\'atil}&\multirow{4}{5.2cm}{Aplicaci\'on limitada \\ Lento \\ Limite detecci\'on limitado} \\
 & & & \\
 & & & \\
 & & & \\
\hline
\multirow{4}{2.2cm}{Absorci\'on at\'omica con generaci\'on hidruros}&\multirow{4}{2cm}{Se, Te, Bi, Sb, As,}&\multirow{4}{2.5cm}{Bajo limite detecci\'on}&\multirow{4}{5.2cm}{Inversi\'on complementaria \\ Aplicaci\'on limitada \\Lento} \\
 & & & \\
 & & & \\
 & & & \\
\hline
\multirow{4}{2.2cm}{Absorci\'on at\'omica con generaci\'on hidruros}&\multirow{4}{2cm}{Se, Te, Bi, Sb, As,}&\multirow{4}{2.5cm}{Bajo limite detecci\'on}&\multirow{4}{5.2cm}{Inversi\'on complementaria \\ Aplicaci\'on limitada \\Lento} \\
 & & & \\
 & & & \\
 & & & \\
\hline
\multirow{4}{2.2cm}{Plasma inductivo (ICP)}&\multirow{4}{2cm}{As, Sb, Fe, Ni, Pb, Bi, Te, Zn, Sn, Cd, Ag}&\multirow{4}{2.5cm}{Bajo limite detecci\'on}&\multirow{4}{5.2cm}{Alta inversi\'on \\Lenta preparaci\'on de muestras} \\
 & & & \\
 & & & \\
 & & & \\
\hline
\multirow{2}{2.2cm}{Combusti\'on}&\multirow{2}{2cm}{S, O}&\multirow{2}{2.5cm}{Rapidez y alta precisi\'on}&\multirow{2}{5.2cm}{Alta inversi\'on. Aplicaci\'on limitada o exclusiva} \\
 & & & \\
\hline
\multirow{2}{2.2cm}{Electrogravi- \\ metr\'ia}&\multirow{2}{2cm}{Cu}&\multirow{2}{2.5cm}{Alta precisi\'on}&\multirow{2}{5.2cm}{Alta inversi\'on \\ Proceso lento} \\
 & & & \\
\hline
\end{tabular}
\end{center}
\caption{Ventajas  y desventajas de las metodolog\'ias}
\end{table}

\subsection{CERTIFICACI\'ON DE CALIDAD}

En algunos casos las empresas productoras de c\'atodos establecen instalaciones de laboratorios en sus plantas, con el objetivo principal de controlar sus operaciones y  procesos, encargando la certificaci\'on de la calidad de sus productos a una empresa de servicios. El hecho de que un tercero certifique la calidad permite a las empresas productoras proyectar imparcialidad y solvencia ante los potenciales compradores.\\

Esta pr\'actica garantiza por un lado una menor inversi\'on y dotaci\'on de  personal de laboratorio y un adecuado control de procesos, y por otro lado durante la puesta en marcha y primeras ventas, ofrecer un producto que ha sido certificado por un laboratorio de reconocido prestigio, con lo cual la penetraci\'on y aceptaci\'on del mercado podr\'ia ser r\'apida y de buena acogida.\\

Esta pr\'actica encierra un riesgo, que el laboratorio cometa errores y no los reconozca, con lo cual induce al productor a entrar en discusi\'on o negociaciones con el comprador a un costo no despreciable. Esto puede ser evitado si existen instancias de control del prestador del servicio ya sea internas o externas, la auditoria peri\'odica y controles de trazabilidad.\\

Lo anterior es razonable, sin embargo para que esto se cumpla seg\'un lo planeado, lo importante y fundamental, es que por una parte la muestra sea representativa y  por otra, que la empresa que efect\'ue la certificaci\'on cuente con los recursos adecuados en infraestructura, experiencia, equipamiento y profesionales que garanticen que sus resultados ser\'an de una precisi\'on y exactitud tal, que ser\'an reproducibles en los embarques analizados en destino.\\

En los casos que se generan discrepancias entre comprador y vendedor, se recurre al arbitraje y en tal situaci\'on la misi\'on de un representante, es verificar que se cumplan los procedimientos establecidos en las normas ASTM o BS bajo la cual fue calificado el lote, de modo que los resultados obtenidos por el arbitraje sean ajustados a normas.\\

Considerando  estos  antecedentes,  si  el  muestreo  en  origen  es  bueno  y  la caracterizaci\'on anal\'itica son reproducibles en cualquier laboratorio internacional o arbitral, ya sean  basados en las normas ASTM o BS, no es relevante disponer de la certificaci\'on en origen por una empresa con renombre internacional y con representaci\'on en diferentes lugares del mundo, ya que lo que importa es su real capacidad y su nivel t\'ecnico para tomar muestras y efectuar an\'alisis que sean repetibles frente al arbitraje.\\

En Per\'u existe SGS DEL PER\'U que certifica la calidad con la norma ISO 9001.\\

A nivel mundial, esta situaci\'on es semejante, observ\'andose que mayoritariamente se muestrea por taladrado y se analiza por espectrometr\'ia o absorci\'on at\'omica la mayor\'ia de los elementos y el resto por m\'etodos cl\'asicos e instrumentales.

\begin{figure}[H]
 \centering
 \includegraphics[scale=0.5]{Imagenes/Vectorial/tal.png}
 % tal.png: 755x569 pixel, 72dpi, 26.63x20.07 cm, bb=0 0 755 569
 \caption{Tratamiento y an\'alisis de muestras por taladrado o aserrado de c\'atodos.}
 \label{fig:tal}
\end{figure}

\begin{figure}[H]
 \centering
 \includegraphics[scale=0.5]{Imagenes/Vectorial/pun.png}
 % pun.png: 739x562 pixel, 72dpi, 26.07x19.83 cm, bb=0 0 739 562
 \caption{Tratamiento y an\'alisis de muestras por punzonado de c\'atodos.}
 \label{fig:pun}
\end{figure}

\section{MARCO TEORICO DE DISE\~NOS EXPERIMENTALES}

En muchas industrias el uso efectivo del dise\~no de experimentos es la clave para obtener altos rendimientos, reducir la variabilidad, reducir los tiempos de entrega, mejorar los productos, reducir los tiempos de desarrollo de nuevos productos y tener clientes más satisfechos.\\

Un dise\~no de experimentos es una prueba o serie de pruebas en las cuales se hacen cambios a prop\'osito en las variables de entrada de un proceso, de tal forma que se puedan observar e identificar cambios en la respuesta de salida.\\
Los productos resultantes tienen una o m\'as caracter\'isticas de calidad observables o respuestas (Cr\'iticas para la calidad si el cliente reclama por su no cumplimiento - CTQ’s). Algunas de las variables del proceso $X_1$, $X_2$, $X_3$,..., $X_p$ son controlables o factores de control, mientras que otras $Z_1$, $Z_2$, $Z_3$, ..., $Z_q$ no son controlables (a pesar de que pueden ser controladas durante el desarrollo de las pruebas), y se denominan factores de ruido. \\

Los objetivos del dise\~no de experimentos son:

\begin{itemize}
 \item[1.] Determinar cu\'ales variables tienen m\'as influencia en la respuesta, y.
 \item[2.] Determinar en donde ajustar las variables de influencia \'xs, de tal forma que y se acerque al requerimiento nominal deseado.
 \item[3.] Determinar donde ajustar las variables de influencia \'xs de tal forma que la variabilidad en y sea peque\~na.
 \item[4.] Determinar donde ajustar las variables de influencia \'xs de tal forma que los efectos de las variables incontrolables z sean minimizados.
\end{itemize}

Con la aplicaci\'on del DOE durante el desarrollo de los procesos podemos obtener los beneficios siguientes:
\begin{itemize}
 \item[1.] Rendimiento mejorado.
 \item[2.] Variabilidad reducida y comportamiento cercano al valor nominal.
 \item[3.] Tiempo de desarrollo reducido.
 \item[4.] Costos totales reducidos.
 \item[5.] Mejor desempe\~no y confiabilidad en el campo.
\end{itemize}

Como ejemplos de aplicaciones del dise\~no de experimentos tenemos las siguientes: 
\begin{itemize}
 \item[1.] Evaluaci\'on y comparaci\'on de configuraciones b\'asicas de dise\~no.
 \item[2.] Evaluaci\'on de alternativas de material.
 \item[3.] Evaluaci\'on de diferentes proveedores.
 \item[4.] Determinaci\'on de par\'ametros clave de dise\~no (ej: \'angulo, velocidad, m\'etodo) con impacto en el desempe\~no.
\end{itemize}

\subsection{GU\'IA PARA EL DISE\~NO DE EXPERIMENTOS} 

Para tener \'exito en el dise\~no de experimentos, es necesario que todos los involucrados en el experimento tengan una idea clara del objetivo del experimento, de los factores a ser estudiados, como se realizar\'a el experimento y al menos una idea cualitativa de c\'omo se analizar\'an los datos.

El procedimiento recomendado por Montgomery tiene los pasos siguientes:
\begin{enumerate}
 \item Reconocimiento y establecimiento del problema.
 \item Selecci\'on de factores y niveles.
 \item Selecci\'on de la variable de respuesta.
 \item Selecci\'on del dise\~no experimental.
 \item Realizaci\'on del experimento.
 \item An\'alisis de los datos.
 \item Conclusiones y recomendaciones.
\end{enumerate}


