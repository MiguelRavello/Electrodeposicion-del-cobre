%%%%%%%%%%%%%%%%%%%%%%%%%%%%%%%%%%%%%%%%%%%%%%%%%%%%%%%%%%%%%%%%%%%%%%%%%%%%%%%%%%%%%%%%%%%%%%
% Plantilla para tesis por David Luna se encuentra bajo una Licencia 
% Creative Commons Atribución-NoComercial-CompartirIgual 3.0 Unported.
% http://creativecommons.org/licenses/by-nc-sa/3.0/
%%%%%%%%%%%%%%%%%%%%%%%%%%%%%%%%%%%%%%%%%%%%%%%%%%%%%%%%%%%%%%%%%%%%%%%%%%%%%%%%%%%%%%%%%%%%%%
\chapter{CONSIDERACIONES GENERALES}
\section{ANTECEDENTES}
El contenido de impurezas en los c\'atodos de cobre, puede representar para las empresas productoras un fuerte castigo en los precios con el consiguiente perjuicio econ\'omico.

Para el control de dichas impurezas, se han desarrollado m\'ultiples t\'ecnicas y normativas, sin embargo, las empresas productoras, a pesar de cumplir en gran medida con determinados tipos de controles, no pueden evitar que se produzcan lotes de baja calidad, los que deben ser detectados oportunamente para tomar las acciones correctivas en el proceso.
Entonces, este estudio se desarroll\'o con el prop\'osito de identificar las variables que tienen mayor incidencia en la producci\'on de c\'atodos de cobre de alta calidad, espec\'ificamente se hizo hincapi\'e en los porcentajes de plomo y azufre que son las impurezas relevantes en la discriminaci\'on de c\'atodos de alta calidad, considerando la Norma L.M.E (London Metal Exchange), que rige el mercado del cobre.
El experimento fue desarrollado utilizando datos representativos de las condiciones actuales de operaci\'on bajo un enfoque de dise\~no experimental con dos niveles y con tres factores para el an\'alisis del plomo y de seis factores para el caso del azufre.
Los resultados obtenidos ofrecen un importante referente para orientar futuras acciones de mejoramiento de la calidad y productividad as\'i como poder identificar factores metal\'urgicos que poseen un fundamento estad\'istico objetivo y, adem\'as, contar con ecuaciones preliminares para estimar porcentajes de impurezas ante eventuales cambios en las condiciones de operaci\'on.

\section{\'AREA QUE SE CIRCUNSCRIBE} 

Ingenier\'ia metal\'urgica, electro obtenci\'on de cobre. Dise\~nos experimentales.  Optimizaci\'on de procesos.

\section{FUNDAMENTOS DE LA INVESTIGACI\'ON}

\subsection{DEFINICI\'ON DEL PROBLEMA}
Los c\'atodos obtenidos por procesos de extracci\'on por solventes y Electro obtenci\'on (SX-EW), son en general de muy alta calidad y \'esta puede ser afectada normalmente por muy pocos elementos, entre los cuales, los más importantes son el plomo, azufre y cloro. 
El plomo se produce por la corrosi\'on del \'anodo de plomo que se emplea en la electro obtenci\'on del cobre; y el azufre es un contaminante com\'un del agua, medio ambiente, minerales y en nuestro caso particular de cuando retiramos la película protectora del alambre de cobre con fuego.
Es motivo del presente estudio analizar la problem\'atica de los contenidos de azufre y plomo que son las principales impurezas e los c\'atodos.

\subsection{OBJETIVOS}

\subsubsection{OBJETIVO GENERAL}
Asegurar la calidad de los c\'atodos manteniendo la impurezas de Plomo y Azufre en el menor valor posible cumpliendo los Est\'andares internacionales de calidad determinas por LME.

\subsubsection{OBJETIVOS ESPEC\'IFICOS}
\begin{itemize}
 \item Identificaremos los factores que afectan de manera directa al proceso de electro obtenci\'on y que inciden de manera directa en el contenido de azufre y plomo en los c\'atodos.
 \item Propondremos valores \'optimos de trabajo para lograr menores cantidades de impurezas en nuestros c\'atodos.
\end{itemize}

\subsection{HIP\'OTESIS}
Al modificar y controlar la temperatura del electrolito, densidad de corriente y concentraci\'on de cobalto  se puede disminuir el contenido de impurezas de plomo en el c\'atodo.
As\'i mismo al modificar y controlar densidad de corriente, concentraci\'on de Cu, temperatura del electrolito y la cantidad de goma guar se puede disminuir el contenido de impurezas de azufre en el c\'atodo.

\subsection{JUSTIFICACI\'ON}

\subsubsection{JUSTIFICACI\'ON TECNOL\'OGICA}
Actualmente los c\'atodos de cobre presentan un promedio de 3 ppm de plomo y de 8 ppm de azufre como límites m\'aximos permisibles, sin embargo dichos contenidos pueden ser reducidos  con un mejor control. Por lo que es necesario determinar los par\'ametros \'optimos de las variables temperatura del electrolito, densidad de corriente, concentraci\'on de cobalto, concentraci\'on de Cu y cantidad de goma guar. Al momento no se han realizado estudios detallados al respecto por lo que el presente trabajo contribuye de manera significativa en mejorar la calidad de los c\'atodos de cobre.

\subsubsection{JUSTIFICACI\'ON ECON\'OMICA}
La comercializaci\'on de los c\'atodos de cobre de GRADO A-LME, dependen exclusivamente de la calidad de los mismos y por consiguiente de la certificaci\'on que se logre para dicho producto manteniendo los m\'as altos est\'andares de calidad de impurezas. Sucede que cuando un comprador detecta presencia de impurezas a niveles fuera de la norma, se muestra renuente a la compra del mismo e inclusive se devuelven lotes completos del producto por estas impurezas.

\subsubsection{JUSTIFICACI\'ON SOCIAL}
La mejora en est\'andares de calidad en los c\'atodos de cobre, incide en la mejora sustancial en las utilidades de las empresas y por tanto estas generan m\'as puestos de trabajo que beneficiaran a las personas que se encuentran en zonas de influencia de las minas. 

\subsection{ALCANCES DEL ESTUDIO}
El presente trabajo de investigaci\'on es una primera etapa para optimizar el proceso de obtenci\'on de c\'atodos de calidad m\'axima y constante en el tiempo. Por ello para eliminar el efecto de aquellas variables aleatorias, es que se realiza la investigaci\'on con un electrolito sint\'etico. En una segunda etapa, una vez comprendido y cuantificado el efecto puro de las variables principales, se deber\'a continuar el estudio a\~nadiendo algunos interferentes al electrolito y finalmente utilizar un electrolito de planta. 
