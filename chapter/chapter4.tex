\chapter{AN\'ALISIS DE LOS RESULTADOS DE LA INVESTIGACION}

\section{AN\'ALISIS E INTERPRETACI\'ON DE LOS RESULTADOS}

Para los an\'alisis efectuados para el PLOMO decimos que: \\

De la tabla ANOVA para el PLOMO podemos decir que al 95\% de confianza solo hay dos elementos importantes para nuestro proceso (p-valores menores que 0.05) y resultan ser las interacciones de: Temperatura VS Densidad y Cobalto VS Densidad.\\

Del gr\'afico de pareto y el de efectos principales podemos afirmar que podemos obtener menores contenidos de plomo cuando tenemos cuidado con la temperatura a $44^oC$, con una densidad de corriente de $315A/m^2$ y con contenidos de $160gr/m^3$ de cobalto.\\

De la gr\'afica de interacciones podemos decir que las interacciones de: Temperatura VS Densidad y Cobalto VS Densidad son relevantes para nuestro proceso y debemos de tratar de controlarlas al un\'isono dichas variables.\\

Del gr\'afico de Temperatura VS Densidad podemos decir que podemos obtener los menores contenidos de plomo debemos mantener la temperatura entre 47 y 48 grados cent\'igrados y a su vez la densidad de corriente entre 260 y 265$amp/cm^2$.\\

Del gr\'afico de Cobalto VS Densidad podemos decir que para obtener menores contenidos de plomo debemos mantener el sulfato de cobalto entre 130 y 140$gr/m^3$ y a su vez la densidad de corriente entre 260 y 270$amp/cm^2$.

Para los an\'alisis efectuados para el AZUFRE decimos que:\\

De la tabla ANOVA podemos decir que al 95\% de confianza hay 3 variables principales que afectan a nuestro proceso en mayor medida: el contenido de Cobre, la Densidad de corriente y la Goma Guar.\\

De los gr\'aficos de pareto confirmamos lo obtenido en la tabla de an\'alisis de la varianza y decimos que son las variables m\'as importantes el Cobre, la Densidad de corriente y la Goma Guar.\\

De los gr\'aficos de superficie de respuesta estimada:
\begin{itemize}
 \item Que en la relaci\'on densidad de corriente vs cobre hay que mantener valores altos de corriente y cobre para bajar el contenido de azufre.
 \item Que en la relaci\'on cobre vs goma guar hay que mantener valores altos de goma guar y cobre para bajar el contenido de azufre.
 \item Que en la relaci\'on densidad de corriente vs goma guar hay que mantener valores altos de corriente y goma guar para bajar el contenido de azufre.
\end{itemize}

\section{VALIDACI\'ON DE LA HIPOTESIS}
De las tablas de respuesta optimizada obtenidas al utilizar las ecuaciones y ejecutarlas de manera iterativa con los valores de referencias se obtuvieron los siguientes datos:

Meta: minimizar Azufre \phantom{eres un pendejo} Valor Optimo = 5.39314

\begin{table}[H]
\label{tabla41}
\begin{center}
\begin{tabular}{|c|c|c|c|}
\hline
Factor&Inferior&Mayor&\'Optimo\\
\hline
Densidad corriente&260.0&315.0&315.0\\  
\hline
Temperatura&44.0&48.0&48.0\\
\hline
Cobre&40.7&44.8&44.8\\
\hline
Goma guar&90.0&180.0&180.0 \\
\hline
\end{tabular}
\end{center}
\caption{Respuesta optimizada}
\end{table}

Meta: minimizar Plomo \phantom{eres un pendejo} Valor Optimo = 2.70314

\begin{table}[H]
\label{tabla42}
\begin{center}
\begin{tabular}{|c|c|c|c|}
\hline
Factor&Inferior&Mayor&\'Optimo\\
\hline
Temperatura&44.0&48.0&44.0\\
\hline
Densidad corriente&260.0&315.0&315.0\\
\hline
Cobalto&135.0&160.0&160.0\\
\hline
\end{tabular}
\end{center}
\caption{Respuesta optimizada}
\end{table}

Vemos que al controlar las variables que afectan el proceso de electro obtenci\'on, podremos disminuir \'o mantener bajo el contenido de impurezas de azufre y plomo en nuestro c\'atodo.\\
Luego de haber realizado la parte experimental se lleg\'o a los siguientes par\'ametros \'optimos de trabajo para minimizar el contenido de impurezas de azufre y plomo:

\begin{table}[H]
\label{tabla42}
\begin{center}
\begin{tabular}{|c|c|c|c|}
\hline
\multirow{2}{5cm}{\phantom{baphomet} VARIABLE}&\multirow{2}{2cm}{AZUFRE}&\multirow{2}{2cm}{PLOMO}&\multirow{2}{2cm}{VALORES\\ \'OPTIMOS}\\
 & & & \\
\hline
Cobre en soluci\'on (gpl)&	44.8&	---&44.8 \\
Temperatura $(^oC)$&	48&	44&	44 - 48(*) \\
Densidad de corriente $(A/m^2)$	&315	&315&	315\\
Goma guar $(gr/m^3)$	&180	&--- &	180\\
Cobalto (ppm)&	---	&160	&160\\
\hline
\end{tabular}
\end{center}
\caption{Par\'ametros \'optimos de trabajo para minimizar el contenido de impurezas de azufre y plomo}
\end{table}
(*) Se sugiere profundizar en estudios posteriores acerca de este dato ya que si bien es cierto se puede incrementar la cin\'etica a mayor temperatura, acarrea un costo elevado para mantener dicha temperatura.

