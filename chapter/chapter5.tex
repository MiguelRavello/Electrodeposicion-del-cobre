\chapter{CONCLUSIONES}

\begin{enumerate}
 \item Se determin\'o que las variables que m\'as inciden en el contenido de plomo son la temperatura del electrolito, densidad de corriente y concentraci\'on de cobalto; sin embargo no de manera independiente sino como interacciones entre cobalto-densidad de corriente y temperatura-densidad de corriente.
 \item Se debe mantener la temperatura entre 47 y 48 grados cent\'igrados y a su vez la densidad de corriente entre 260 y 265$amp/cm^2$ y el sulfato de cobalto entre 130 y 140$gr/m^3$ y a su vez la densidad de corriente entre 260 y 270$amp/cm^2$.
 \item  Para el azufre se determin\'o que las variables que m\'as importan en nuestro proceso son el contenido de cobre en soluci\'on, la densidad de corriente y la dosificaci\'on de Goma guar. La goma guar debe mantenerse entre los 150 y 190$gr/m^3$, la densidad de corriente entre 290 y 320$amp/m^2$ y el cobre debe estar por encima de los 43 gpl a 45 gpl.
\end{enumerate}
