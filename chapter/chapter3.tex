\chapter{MATERIALES Y M\'ETODOS}

\section{PRUEBAS EXPERIMENTALES}
\subsection{EQUIPOS, MATERIALES Y REACTIVOS QU\'IMICOS}
\begin{itemize}
 \item Rectificador de corriente con capacidad de transformaci\'on de 30 amperes y 110 volts. (Para transformar la energ\'ia el\'ectrica alterna en corriente directa); la fuente de poder se alimenta con energ\'ia el\'ectrica alterna de 110 volts.

 \item Celda Electrolítica. La celda electrol\'itica se construye de dimensiones 15x15x30cm, cuatro \'anodos de plomo y tres c\'atodos de acero inoxidable de 10cm. de ancho y 10cm. de largo.

 \item Soluci\'on (Electr\'olito) tiene concentraciones de cobre y \'acido sulf\'urico de $45$ y $180 \frac{gr}{L}$ respectivamente; el cobre est\'a ionizado en forma de sulfato de cobre. 

 \item Balanza anal\'itica de plato de 50 gramos de capacidad con 0.001 gramos de error.

 \item Espectrofot\'ometro de absorci\'on at\'omica para analizar muestras de los c\'atodos depositado en an\'alisis de plomo y azufre (se contrato el servicio de un laboratorio particular) con l\'amparas de cobalto, cobre, plomo y azufre alimentado con acetileno y ox\'igeno para llegar a un rango promedio de $3000^oC$ de llama y que trabaja por digesti\'on acida con una sensibilidad de ppb (partes por bill\'on \'o miligramos por tonelada).

 \item \'Acido sulf\'urico qu\'imicamente puro, las diluciones ser\'an preparadas a partir de este para obtener el electrolito sint\'etico.

 \item Material de vidrio: 
 \item Vasos de precipitados de 50, 100 ,1000$cm^3$.
 \item Picetas de 500$cm^3$.
 \item Pipetas de 2 y 10$cm^3$.
 \item Matraces de 50$cm^3$.
 \item Probeta graduada de 10$cm^3$.
 \item Vidrio de reloj
 \item Varilla de vidrio
 \item Embudo de vidrio
 \item Lentes de seguridad.
 \item Mandil de laboratorio anti\'acido.
 \item Guantes de nitrilo y neopreno para el manejo del acido QP.
 \item Respirador de silicona con cartuchos de gas org\'anico.
 \item Term\'ometro digital.
 \item Estufa de laboratorio para calentar soluciones.
\end{itemize}

\subsection{PROCEDIMIENTO EXPERIMENTAL}
Inicialmente se procedi\'o a preparar el electrolito sint\'etico:

\begin{center}
 \begin{tabular}{|c|c|}
 \hline
 \multicolumn{2}{|c|}{Electrolito sint\'etico}\\
 \hline
 Cu& $40.7-44.8\:gpl$(seg\'un experimentos)\\
 \hline
 Acido&$170gpl$\\
 \hline
\end{tabular}
\end{center} 


Esto quiere decir por ejemplo que para una soluci\'on de 40.7 gramos por litro de cobre y 170 gramos por litro de \'acido se tuvo que  agregar $92.69cm^3$ de \'acido (siendo su densidad $1.834\frac{gr}{ml}$) y enrasar a un litro de soluci\'on en un vaso de precipitado de 1 litro de capacidad. \\
Luego agregamos los trozos de alambre de cobre a los que aparte de pelar, se les quem\'o para eliminar la película de recubrimiento que llevan consigo, se agreg\'o el peso de 40.7 gramos y se dej\'o disolviendo en la soluci\'on acida hasta su disoluci\'on total; obteniendo as\'i un electrolito sint\'etico de acuerdo a nuestra necesidades. \\
Luego de lo cual se procedi\'o a calentar las soluciones a las temperaturas que vamos a usar como variables de control.
Entonces se procedi\'o a los experimentos de acuerdo a siguiente secuencia :
\begin{enumerate}
 \item Pesar los c\'atodos (l\'aminas de acero inoxidable) 
 \item  Colocar 2L de soluci\'on en la celda, introducir los electrodos (\'anodos y c\'atodos) y conectarlos a la fuente de poder, c\'atodo al negativo y \'anodo al positivo, adem\'as de agregar los aditivos que se requieren de acuerdo a cada experimento. 
 \item  Accionar el interruptor de la fuente y se inicia la electrodeposici\'on del cobre. 
 \item  Proporcionar un tiempo de electro obtenci\'on de 1 hora. 
 \item  Interrumpir la energ\'ia el\'ectrica y extraer el c\'atodo con el cobre depositado, lavarlo con agua. 
 \item  Secar el c\'atodo y pesarlo, la diferencia en peso es el cobre depositado en una hora. 
 \item  Muestrear de los c\'atodos electro obtenidos y se enviar\'an para su an\'alisis por espectrofotometr\'ia de absorci\'on at\'omica.
\end{enumerate}

\subsection{DETERMINACI\'ON DE VARIABLES DE CONTROL}
Para identificar las variables relevantes que inciden en la calidad del cobre, se identificaron los factores que inciden directamente en los niveles de plomo y azufre presentes en los c\'atodos de cobre.\\
Los factores considerados en un principio fueron los siguientes:
Concentraci\'on de Cobre, Concentraci\'on de Cobalto, Temperatura del electrolito, Densidad de Corriente, Peso C\'atodos, Dosificaci\'on de Guar y Flujo Alimentaci\'on Celda.\\
Dada la experiencia operacional previa existente sobre la materia se opt\'o por realizar dos dise\~nos experimentales utilizando la informaci\'on hist\'orica ya existente y previamente depurada. El car\'acter de los dise\~nos fue el siguiente: un dise\~no $2^3$ con el plomo como su variable de respuesta y un segundo dise\~no $2^{4-1}$ con el azufre como su variable de respuesta.

\subsubsection{VARIABLES INDEPENDIENTES}
\begin{itemize}
 \item Plomo (ppm)
 \item Azufre (ppm)
\end{itemize}

\subsubsection{VARIABLES DEPENDIENTES}

Para el caso puntual de las variables de control o variables dependientes utilizaremos la variables m\'as ampliamente usadas en referencias bibliogr\'aficas y a trav\'es de los experimentos que realizaremos, determinaremos en que dimensi\'on son efectivamente importantes para nuestro proceso o si de hecho no lo son las descartaremos a medida que avance nuestro trabajo de investigaci\'on.

\begin{table}[H]
\label{tabla11}
\begin{center}
\begin{tabular}{|c|c|c|c|}
\hline
VARIABLES&MINIMO(-1)&MAXIMO(+1)\\
\hline
Temperatura de electrolito ($^oC$)&44&48\\
\hline
Densidad de corriente ($\frac{A}{m^2}$)&260&315\\
\hline
Concentraci\'on de cobalto (ppm)&135&160\\
\hline
\end{tabular}
\end{center}
\caption{Variables dependientes para el plomo}
\end{table}

\begin{table}[H]
\label{tabla12}
\begin{center}
\begin{tabular}{|c|c|c|c|}
\hline
VARIABLES&MINIMO(-1)&MAXIMO(+1)\\
\hline
Densidad de corriente ($\frac{A}{m^2}$)&260&315\\
\hline
Cu (gpl)&40.7&44.8\\
\hline
Temperatura de electrolito ($^oC$)&44&48\\
\hline
Goma guar ($\frac{gr}{tm}$)&90&180\\
\hline
\end{tabular}
\end{center}
\caption{Variables dependientes para el azufre}
\end{table}

\subsection{C\'ALCULOS}

Por ejemplo para el experimento n\'umero 1 del Plomo se emple\'o un electrolito de $42.75gpl$ de cobre (el promedio de los valores propuestos ya que no es variable de control para el plomo), $170gpl$ de \'acido sulf\'urico ($92.69cm^3$ ya que su densidad es $1.834\frac{gr}{cm^3}$), se emple\'o una temperatura de 46 grados cent\'igrados, una densidad de corriente de 17.25 amperes (ya que las dimensiones de los c\'atodos son de 0.1mx0.1m su \'area es de 0.01$m^2$ a ello se multiplica por las 2 caras y por los 3 c\'atodos y se multiplican por el valor referencial de la prueba que es de 287.5 amperes por $m^2$), asimismo se agrega 0.295 gramos de sulfato de cobalto (en vista de que son 147.5 $\frac{gr}{m^3}$ de soluci\'on, $1m^3$ tiene 1000 litros y nosotros utilizamos 2 litros por celda) y la aplicaci\'on de 2 voltios por cada celda.\\
Y as\'i sucesivamente vamos planteando y ejecutando cada uno de los 11 experimentos propuestos a los que a continuaci\'on mostramos los resultados obtenidos:

\begin{table}[H]
\label{tabla13}
\begin{center}
\begin{tabular}{|c|c|c|c|c|}
\hline
\multirow{2}{0.5cm}{$N^o$}&\multirow{2}{1.8cm}{Tempera- \\tura($^oC$)}&\multirow{2}{2.9cm}{Densidad \\ corriente $(A/m^2)$}&\multirow{2}{1.7cm}{Cobalto \\(ppm)}&\multirow{2}{1.7cm}{Plomo \\ (ppm)}\\
 & & & & \\
\hline 
1&	46&	287.5&	147.5&	3.266\\ 
\hline
2&	44&	260&	135&	3.169\\
\hline
3&	48&	260&	135&	2.997\\
\hline
4&	44&	315&	135&	4.276\\
\hline
5&	48&	315&	135&	5.544\\
\hline
6&	46&	287.5&	147.5&	3.682\\
\hline
7&	44&	260&	160&	4.549\\
\hline
8&	48&	260&	160&	3.175\\
\hline
9&	44&	315&	160&	2.621\\
\hline
10&	48&	315&	160&	3.842\\
\hline
11&	46&	287.5&	147.5&	3.680\\
\hline
\end{tabular}
\end{center}
\caption{Experimentos realizados en laboratorio y resultados obtenidos para el plomo}
\end{table}

Asimismo para el caso del azufre se procedi\'o a evaluar los siguientes experimentos.\\

Por ejemplo para el experimento n\'umero 1 del azufre se emple\'o un electrolito de $42.75gpl$ de cobre (valor propuesto como variable de control), $170gpl$ de \'acido sulf\'urico ($92.69cm^3$ ya que su densidad es $1.834 gr/cm^3$), se emple\'o una temperatura de 46 grados cent\'igrados, una densidad de corriente de 17.25 amperes (ya que las dimensiones de los c\'atodos son de 0.1mx0.1m su \'area es de 0.01$m^2$ a ello se multiplica por las 2 caras y por los 3 c\'atodos y se multiplican por el valor referencial de la prueba que es de 287.5 amperes por $m^2$), asimismo se agrega 0.27 gramos de goma guar (en vista de que son $135 gr/m^3$ de soluci\'on, $1m^3$ tiene 1000 litros y nosotros utilizamos 2 litros por celda) y la aplicaci\'on de 2 voltios por cada celda.

Y as\'i sucesivamente vamos planteando y ejecutando cada uno de los 11 experimentos propuestos a los que a continuaci\'on mostramos los resultados obtenidos:

\begin{table}[H]
\label{tabla14}
\begin{center}
\begin{tabular}{|c|c|c|c|c|c|}
\hline
\multirow{2}{0.5cm}{$N^o$}&\multirow{2}{2.7cm}{Densidad\\ corriente$(A/m^2)$}&\multirow{2}{1.8cm}{Tempera- \\tura($^oC$)}&\multirow{2}{1cm}{Cobre \\(gpl)}&\multirow{2}{1.9cm}{Goma guar $(gr/m^3)$}&\multirow{2}{1.42cm}{Plomo \\ (ppm)}\\
 & & & & & \\
\hline 
1&	287.5&	46&	42.75&	135&	7.058\\
\hline
2&	260&	44&	40.7&	90&	8.973\\
\hline
3&	315&	44&	40.7&	180&	7.961\\
\hline
4&	260&	48&	40.7&	180&	7.991\\
\hline
5&	315&	48&	40.7&	90&	8.150\\
\hline
6&	287.5&	46&	42.75&	135&	7.059\\
\hline
7&	260&	44&	44.8&	180&	6.173\\
\hline
8&	315&	44&	44.8&	90&	5.999\\
\hline
9&	260&	48&	44.8&	90&	8.066\\
\hline
10&	315&	48&	44.8&	180&	5.472\\
\hline
11&	287.5&	46&	42.75&	135&	7.064\\
\hline
\end{tabular}
\end{center}
\caption{Experimentos realizados en laboratorio y resultados obtenidos para el azufre}
\end{table}

\subsection{TRATAMIENTO DE RESULTADOS}
Es entonces que procedemos al an\'alisis por computador apoy\'andonos en el programa estad\'istico STATGRAPHICS, es necesario mencionar que dicho an\'alisis se puede ejecutar en cualquier software estad\'istico y cuyos resultados son:

\begin{figure}[H]
 \centering
 \includegraphics[scale=0.25]{Imagenes/Vectorial/stat.png}
 % stat.png: 1280x800 pixel, 72dpi, 45.16x28.22 cm, bb=0 0 1280 800
 \caption{An\'alisis de la varianza (anova) para el plomo}
 \label{fig:sta}
\end{figure}

El indicador r-cuadrado nos indica al 94.23\% de confianza que nuestro an\'alisis es consistente; adem\'as el valor de la columna f-ratio nos indica el orden de importancia de nuestra variables analizadas, siendo que la interacci\'on densidad-cobalto es la m\'as importante (f-ratio de 30.48) y adem\'as la columna del p-valor es la probabilidad de que nuestra variables sean importantes en el dise\~no y siendo que nuestra confianza al analizar es de 95\% (es decir 0.95) el error es de 0.05 (5\% restante) y son precisamente los valores de esta columna inferiores a 0.05 los que representan a las variables importantes en nuestro dise\~no siendo las interacciones densidad-cobalto y temperatura-densidad.\\

Asimismo se obtuvo una ecuaci\'on que representa la tendencia de las variables:

\begin{equation}
 \begin{align}
Plomo &= 3.64661 - 1.65628*Temperatura - 0.147416*Densidad corriente\\
&+0.783136*Cobalto + 0.00917101*Temperatura*Densidadcorriente\\ &-0.00624726*Temperatura*Cobalto - 0.00178695*Densidad\\ &corriente*Cobalto \notag 
 \end{align}
\end{equation}

Luego con la ecuaci\'on obtenida se procedi\'o a calcular de manera iterativa la concentraci\'on \'optima de plomo (siendo que deseamos disminuirla), obteniendo los siguientes resultados de valores recomendados para las variables de control:\\

Meta: minimizar Plomo \phantom{eres un pendejo} Valor Optimo = 2.70314

\begin{table}[H]
\label{tabla15}
\begin{center}
\begin{tabular}{|c|c|c|c|}
\hline
Factor&Inferior&Mayor&\'Optimo\\
\hline
Temperatura&44.0&48.0&44.0\\
\hline
Densidad corriente&260.0&315.0&315.0\\
\hline
Cobalto&135.0&160.0&160.0\\
\hline
\end{tabular}
\end{center}
\caption{Respuesta optimizada}
\end{table}

En el siguiente gr\'afico de pareto vemos la misma relaci\'on de la tabla anova que indica que las variables m\'as representativas son las interacciones densidad-cobalto y temperatura-densidad.

\begin{figure}[H]
 \centering
 \includegraphics[scale=0.25]{Imagenes/Vectorial/f1.png}
 % f1.png: 1280x800 pixel, 72dpi, 45.16x28.22 cm, bb=0 0 1280 800
 \caption{Gr\'afica de pareto para el plomo}
 \label{fig:f1}
\end{figure}

En la siguiente gr\'afica se observa en qu\'e medida afectan las variables de control el contenido de plomo en ppm en el c\'atodo electroobtenido, siendo que la variable que m\'as importa es la densidad de corriente, aunque estad\'isticamente son m\'as importantes las interacciones que las variables por si solas con aproximadamente 0.6 ppm a medida que disminuye.

\begin{figure}[H]
 \centering
 \includegraphics[scale=0.25]{Imagenes/Vectorial/f2.png}
 % f2.png: 1280x800 pixel, 72dpi, 45.16x28.22 cm, bb=0 0 1280 800
 \caption{Gr\'afica de efectos para el plomo.}
 \label{fig:f2}
\end{figure}

En la gr\'afica siguiente vemos las interacciones importantes (aquellas que se cruzan entre s\'i) siendo las interacciones AB aproximadamente 1.2 ppm y la interacci\'on BC aproximadamente 1.6 ppm respecto del contenido de plomo en el c\'atodo electro obtenido.

\begin{figure}[H]
 \centering
 \includegraphics[scale=0.25]{Imagenes/Vectorial/f3.png}
 % f3.png: 1280x800 pixel, 72dpi, 45.16x28.22 cm, bb=0 0 1280 800
 \caption{Gr\'afica de interacciones para el plomo.}
 \label{fig:f3}
\end{figure}

En la siguiente gr\'afica de 2 dimensiones de la densidad de corriente vs temperatura podemos apreciar que para disminuir el contenido de plomo en nuestros c\'atodos (color rosado de la leyenda de 3.0 a 3.18 ppm de plomo) debemos mantener la temperatura entre 47 y 48 grados cent\'igrados y a su vez la densidad de corriente entre 260 y 265 $amp/cm^2$.

\begin{figure}[H]
 \centering
 \includegraphics[scale=0.25]{Imagenes/Vectorial/f4.png}
 % f4.png: 1280x800 pixel, 72dpi, 45.16x28.22 cm, bb=0 0 1280 800
 \caption{Gr\'afica de superficie de respuesta estimada de Temperatura VS Densidad.}
 \label{fig:f4}
\end{figure}

En la siguiente gr\'afica de 2 dimensiones del cobalto vs densidad de corriente podemos apreciar que para disminuir el contenido de plomo en nuestros c\'atodos (color rosado de la leyenda de 3.0 a 3.18 ppm de plomo) debemos mantener el sulfato de cobalto entre 130 y 140 $gr/m^3$ y a su vez la densidad de corriente entre 260 y 270 $amp/cm^2$.

\begin{figure}[H]
 \centering
 \includegraphics[scale=0.25]{Imagenes/Vectorial/f5.png}
 % f5.png: 1280x800 pixel, 72dpi, 45.16x28.22 cm, bb=0 0 1280 800
 \caption{Gr\'afica de superficie de respuesta estimada de Cobalto VS Densidad.}
 \label{fig:f5}
\end{figure}

Luego para lo relativo al azufre tenemos:\\

Inicialmente veremos el an\'alisis de la varianza en donde reflejara que variables de control son estad\'isticamente importantes y en qu\'e orden de importancia respecto de nuestros experimentos para reducir el contenido de azufre ne  nuestros c\'atodos de cobre.

\begin{figure}[H]
 \centering
 \includegraphics[scale=0.25]{Imagenes/Vectorial/f6.png}
 % f6.png: 1280x800 pixel, 72dpi, 45.16x28.22 cm, bb=0 0 1280 800
 \caption{An\'alisis de la varianza (anova) para el azufre.}
 \label{fig:f6}
\end{figure}

El indicador r-cuadrado nos indica al 98.42\% de confianza que nuestro an\'alisis es consistente; adem\'as el valor de la columna f-ratio nos indica el orden de importancia de nuestra variables analizadas, siendo que la variable cobre es lamas importante (f-ratio de 112.4) y adem\'as la columna del p-valor es la probabilidad de que nuestra variables sean importantes en el dise\~no y siendo que nuestra confianza al analizar es de 95\% (es decir 0.95) el error es de 0.05 (5\% restante) y son precisamente los valores de esta columna inferiores a 0.05 los que representan a las variables importantes en nuestro dise\~no siendo las variables cobre, densidad y goma guar (en este orden espec\'ifico).\\

Asimismo obtuvimos una ecuaci\'on que representa a nuestro proceso:

\begin{equation}
 \begin{align}
  Azufre &= 30.8989 - 0.0164625*Densidad Corriente +\\
         &0.0358798*Temperatura - 0.449137*Cobre - \\
         &0.00997226*Goma Guar
               \notag
 \end{align}
\end{equation}
Luego con la ecuaci\'on obtenida se procedi\'o a calcular de manera iterativa la concentraci\'on \'optima de plomo (siendo que deseamos disminuirla), obteniendo los siguientes resultados de valores recomendados para las variables de control:

Meta: minimizar Azufre \phantom{eres un pendejo} Valor Optimo = 5.39314

\begin{table}[H]
\label{tabla16}
\begin{center}
\begin{tabular}{|c|c|c|c|}
\hline
Factor&Inferior&Mayor&\'Optimo\\
\hline
Densidad corriente&260.0&315.0&315.0\\  
\hline
Temperatura&44.0&48.0&48.0\\
\hline
Cobre&40.7&44.8&44.8\\
\hline
Goma guar&90.0&180.0&180.0 \\
\hline
\end{tabular}
\end{center}
\caption{Respuesta optimizada}
\end{table}

En el siguiente gr\'afico de pareto vemos la misma relaci\'on de la tabla anova que indica que las variables m\'as representativas son el cobre, la densidad de corriente y la goma guar en ese orden espec\'ifico. \\

\begin{figure}[H]
 \centering
 \includegraphics[scale=0.25]{Imagenes/Vectorial/s1.png}
 % s1.png: 1280x800 pixel, 72dpi, 45.16x28.22 cm, bb=0 0 1280 800
 \caption{Gr\'afica de pareto para efectos principales.}
 \label{fig:s1}
\end{figure}

En el siguiente gr\'afico de pareto vemos la misma relaci\'on de la tabla anova que indica que las interacciones de las variables no son representativas de nuestro modelo as\'i como tampoco lo es la variable temperatura.

\begin{figure}[H]
 \centering
 \includegraphics[scale=0.25]{Imagenes/Vectorial/s2.png}
 % s1.png: 1280x800 pixel, 72dpi, 45.16x28.22 cm, bb=0 0 1280 800
 \caption{Gr\'afica de pareto para efectos principales y de segundo orden.}
 \label{fig:s2}
\end{figure}

En las siguientes gr\'aficas de 3 dimensiones se observa que:
\begin{itemize}
 \item Que en la relaci\'on densidad vs cobre hay que mantener valores altos de corriente y cobre para bajar el contenido de azufre.
 \item Que en la relaci\'on cobre vs goma guar hay que mantener valores altos de goma guar y cobre para bajar el contenido de azufre.
 \item Que en la relaci\'on densidad vs goma guar hay que mantener valores altos de corriente y goma guar para bajar el contenido de azufre.
\end{itemize}

\begin{figure}[H]
 \centering
 \includegraphics[scale=0.25]{Imagenes/Vectorial/s3.png}
 % s1.png: 1280x800 pixel, 72dpi, 45.16x28.22 cm, bb=0 0 1280 800
 \caption{Gr\'aficas de superficie de respuesta estimada para el azufre-a.}
 \label{fig:s3}
\end{figure}

\begin{figure}[H]
 \centering
 \includegraphics[scale=0.25]{Imagenes/Vectorial/s4.png}
 % s1.png: 1280x800 pixel, 72dpi, 45.16x28.22 cm, bb=0 0 1280 800
 \caption{Gr\'aficas de superficie de respuesta estimada para el azufre-b.}
 \label{fig:s4}
\end{figure}

\begin{figure}[H]
 \centering
 \includegraphics[scale=0.25]{Imagenes/Vectorial/s5.png}
 % s1.png: 1280x800 pixel, 72dpi, 45.16x28.22 cm, bb=0 0 1280 800
 \caption{Gr\'aficas de superficie de respuesta estimada para el azufre-c.}
 \label{fig:s5}
\end{figure}

En las siguientes gr\'aficas de vs podemos entrar m\'as en detalle de las variables de control:\\

En la gr\'afica de goma guar vs densidad de corriente vemos que para mantener el contenido de azufre bajo (color rosa de la leyenda de 6.6 a 6.72 ppm de azufre) la goma guar debe mantenerse entre los 150 y 190 $gr/m^3$ y la densidad de corriente entre 290 y 320 $amp/m^2$.
En la gr\'afica de goma guar vs cobre vemos que para mantener el contenido de azufre bajo (color rosa de la leyenda de 6.6 a 6.72 ppm de azufre) el cobre debe estar por encima de los 43 gpl a 45 gpl.
En la gr\'afica de densidad de corriente vs cobre vemos que para mantener el contenido de azufre bajo (color rosa de la leyenda de 6.6 a 6.72 ppm de azufre) el cobre debe estar por encima de los 43 gpl a 45 gpl.

\begin{figure}[H]
 \centering
 \includegraphics[scale=0.25]{Imagenes/Vectorial/s6.png}
 % s1.png: 1280x800 pixel, 72dpi, 45.16x28.22 cm, bb=0 0 1280 800
 \caption{Gr\'aficas de contornos de respuesta estimada para el azufre-a.}
 \label{fig:s6}
\end{figure}

\begin{figure}[H]
 \centering
 \includegraphics[scale=0.25]{Imagenes/Vectorial/s7.png}
 % s1.png: 1280x800 pixel, 72dpi, 45.16x28.22 cm, bb=0 0 1280 800
 \caption{Gr\'aficas de contornos de respuesta estimada para el azufre-b.}
 \label{fig:s7}
\end{figure}

\begin{figure}[H]
 \centering
 \includegraphics[scale=0.25]{Imagenes/Vectorial/s8.png}
 % s1.png: 1280x800 pixel, 72dpi, 45.16x28.22 cm, bb=0 0 1280 800
 \caption{Gr\'aficas de contornos de respuesta estimada para el azufre-c.}
 \label{fig:s8}
\end{figure}


