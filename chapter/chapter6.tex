\chapter{RECOMENDACIONES}

\begin{enumerate}
 \item Se sabe que los dise\~nos experimentales factoriales est\'an dando buenos resultados en el campo de la metalurgia y la qu\'imica a nivel de laboratorio, pero hay que ampliar el espectro de an\'alisis con otras t\'ecnicas como los modelos robustos de Taguchi o los modelos rotables multinivel (dise\~nos compuestos) ya sean factoriales o de superficie de respuesta para analizar en profundidad y de manera m\'as detallada las interacciones que detectamos en la presente investigaci\'on.
 \item Asimismo se recomienda hacer nuevos experimentos a fin de determinar con m\'as claridad la influencia especifica de la temperatura que no ha quedado establecida de manera definitiva en el presente trabajo y si bien es cierto increment\'andola se acelera la cin\'etica de las reacciones, esto conlleva costos econ\'omicos adem\'as de que podría contribuir a la contaminaci\'on de los c\'atodos.
 \item  Por \'ultimo se recomienda plantear experimentos futuros incluyendo la totalidad de las variables consignadas en el presente trabajo y adem\'as otras que pudieron no ser tomadas en cuenta en esta ocasi\'on para hacer modelos de pron\'ostico cada vez m\'as complejos a fin de estructurar un ciclo de mejora continua de los m\'etodos de electroobtenci\'on que empleamos para el cobre no solo circunscribi\'endose  los modelos experimentales (estad\'istica) sino ampli\'andose incluso a los modelos probabil\'isticos y determin\'isticos (simulaci\'on de procesos) para hacer modelos que nos lleven a nuevos horizontes en cuanto a la tecnolog\'ia de obtenci\'on de metales.
 \item  La presente investigaci\'on no debe de ser sino m\'as que una referencia para futuros estudios por parte de estudiantes y egresados de ingenier\'ias que deseen ampliar conocimientos en el campo de la qu\'imica aplicada a procesos metal\'urgicos y electroqu\'imicos.
\end{enumerate}
