%---------------------------------------------------------------------
%
%                      resumenManual.tex
%
%---------------------------------------------------------------------
%
% resumenManual.tex
% Copyright 2009 Marco Antonio Gomez-Martin, Pedro Pablo Gomez-Martin
%
% This file belongs to the TeXiS manual, a LaTeX template for writting
% Thesis and other documents. The complete last TeXiS package can
% be obtained from http://gaia.fdi.ucm.es/projects/texis/
%
% Although the TeXiS template itself is distributed under the 
% conditions of the LaTeX Project Public License
% (http://www.latex-project.org/lppl.txt), the manual content
% uses the CC-BY-SA license that stays that you are free:
%
%    - to share & to copy, distribute and transmit the work
%    - to remix and to adapt the work
%
% under the following conditions:
%
%    - Attribution: you must attribute the work in the manner
%      specified by the author or licensor (but not in any way that
%      suggests that they endorse you or your use of the work).
%    - Share Alike: if you alter, transform, or build upon this
%      work, you may distribute the resulting work only under the
%      same, similar or a compatible license.
%
% The complete license is available in
% http://creativecommons.org/licenses/by-sa/3.0/legalcode
%
%---------------------------------------------------------------------
%
% Contiene el cap�tulo del resumen.
%
% Se crea como un cap�tulo sin numeraci�n.
%
%---------------------------------------------------------------------

\chapter{Resumen}
\cabeceraEspecial{Resumen}

El contenido de impurezas en los c�todos de cobre, puede representar para las empresas productoras un fuerte castigo en los precios con el consiguiente perjuicio econ�mico. \\

Para el control de dichas impurezas, se han desarrollado m�ltiples t�cnicas y normativas, sin embargo, las empresas productoras, a pesar de cumplir en gran medida con determinados tipos de controles, no pueden evitar que se produzcan lotes de baja calidad, los que deben ser detectados oportunamente para tomar las acciones correctivas en el proceso. \\

Entonces, este estudio se desarroll� con el prop�sito de identificar las variables que tienen mayor incidencia en la producci�n de c�todos de cobre de alta calidad, espec�ficamente se hizo hincapi� en los porcentajes de plomo y azufre que son las impurezas relevantes en la discriminaci�n de c�todos de alta calidad, considerando la Norma L.M.E (London Metal Exchange), que rige el mercado del cobre. \\

El experimento fue desarrollado utilizando datos obtenidos a nivel de laboratorio, bajo un enfoque de dise�o experimental con dos niveles y con tres factores para el an�lisis del plomo y de seis factores para el caso del azufre; adem�s se analizara el efecto producido por las interacciones de los mismos en el resultado obtenido en cada experimento. \\

Los resultados obtenidos ofrecen un importante referente para orientar futuras acciones de mejoramiento de la calidad y productividad as� como poder identificar factores metal�rgicos que poseen un fundamento estad�stico objetivo y, adem�s, contar con ecuaciones preliminares para estimar porcentajes de impurezas ante eventuales cambios en las condiciones de operaci�n.





\endinput
% Variable local para emacs, para  que encuentre el fichero maestro de
% compilaci�n y funcionen mejor algunas teclas r�pidas de AucTeX
%%%
%%% Local Variables:
%%% mode: latex
%%% TeX-master: "../ManualTeXiS.tex"
%%% End:
