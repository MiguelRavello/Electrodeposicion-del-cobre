\chapter{Presentaci\'on}

\cabeceraEspecial{Presentaci\'on}

El presente trabajo de investigaci\'on pretende ser una contribuci\'on a las t\'ecnicas de control de calidad aplicadas en la industria minera, en este caso particular en lo que a control de calidad en la obtenci\'on de c\'atodos electrol\'iticos de cobre de grado a. \\

El contenido de impurezas en los c\'atodos de cobre, puede representar para las empresas productoras un fuerte castigo en los precios con el consiguiente perjuicio econ\'omico. \\

Este estudio se desarroll\'o con el prop\'osito de identificar las variables que tienen mayor incidencia en la producci\'on de c\'atodos de cobre de alta calidad, espec\'ificamente se hizo hincapi\'e en los porcentajes de plomo y azufre que son las impurezas relevantes en la discriminaci\'on de c\'atodos de alta calidad, considerando la Norma L.M.E. (London Metal Exchange), que rige el mercado del cobre.\\

Los resultados obtenidos ofrecen un importante referente para orientar futuras acciones de mejoramiento de la calidad.

\endinput